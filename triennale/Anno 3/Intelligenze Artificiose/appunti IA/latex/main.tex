\documentclass[a4paper,12 pt]{report}
\usepackage[T1]{fontenc}
\usepackage[utf8]{inputenc}
\usepackage[italian]{babel}
\usepackage{lmodern}
\usepackage{listings}
\usepackage{latexsym}
\usepackage{graphicx}
\usepackage{float}
\usepackage{subcaption}
\usepackage{hyperref}
\usepackage{wrapfig}
\usepackage{fancyhdr}
\usepackage{amsthm}
\usepackage{amsmath}
\usepackage{amssymb}
\usepackage{amsfonts}
\usepackage{cancel}
\usepackage{tcolorbox}


% forza le footnote a stare il più in basso possibile
\usepackage[bottom]{footmisc}

% stile teoremi
\newtheorem{theorem}{Teorema}[section]
\newtheorem{corollary}{Corollario}[theorem]
\newtheorem{lemma}[theorem]{Lemma}
\newtheorem{axch}{Assioma}[section]
\newtheorem{prop}{Proprietà}[section]
\newtheorem{definition}{Definizione}[section]
\newtheorem{limit}{Limitazione}[section]

\renewcommand*{\proofname}{Dimostrazione}

%% STILE LISTINGS

\usepackage{xcolor}

\definecolor{codegreen}{rgb}{0,0.6,0}
\definecolor{codegray}{rgb}{0.5,0.5,0.5}
\definecolor{codepurple}{rgb}{0.58,0,0.82}
\definecolor{backcolour}{rgb}{0.95,0.95,0.92}

 \lstdefinestyle{mystyle}{
     backgroundcolor=\color{backcolour},   
     commentstyle=\color{codegreen},
     keywordstyle=\bfseries\color{black},
     numberstyle=\tiny\color{codegray},
     stringstyle=\color{codepurple},
     basicstyle=\ttfamily\footnotesize,
     breakatwhitespace=false,         
     breaklines=true,                 
     captionpos=b,                    
     keepspaces=true,                 
     numbers=left,                    
     numbersep=5pt,                  
     showspaces=false,                
     showstringspaces=false,
     showtabs=false,                  
     tabsize=2,
     mathescape=true,
     escapeinside={\%*}{*)},
     morekeywords={
        begin,
        end,
        if,
        then,
        else,
        to,
        endif,
        Procedure,
        while,
        do,
        return,
        true,
        false,
        set,
        for,
        repeat,
        until
        foreach},
 }

 \lstset{style=mystyle}

\renewcommand{\lstlistlistingname}{Elenco dei codici}
\renewcommand{\lstlistingname}{Codice}

\newtcolorbox{definition-box}[1]{%
    colback=codepurple!5!white,%
    colframe=codepurple!75!black,%
    fonttitle=\bfseries,%
    title={#1}
}

% mostra le subsubsection nell'indice
\setcounter{tocdepth}{3}
\setcounter{secnumdepth}{3}

% Resetta la numerazione dei chapter quando
% una nuova part viene creata
\makeatletter
\@addtoreset{chapter}{part}
\makeatother

% Rimuove l'indentazione quando si crea un nuovo paragrafo
\setlength{\parindent}{0pt}

% footer
\pagestyle{fancyplain}
% rimuove la riga nell'header
\fancyhf{} % sets both header and footer to nothing
\renewcommand{\headrulewidth}{0pt}
\fancyfoot[L]{\href{https://github.com/Typing-Monkeys/AppuntiUniversita}{Typing Monkeys}}
\fancyfoot[C]{\emoji{gorilla}}
\fancyfoot[R]{\thepage}

% configurazione emoji
\usepackage{fontspec}
\usepackage{emoji}
\setemojifont{NotoColorEmoji.ttf}[Path=/usr/share/fonts/truetype/noto/]

\setlength {\marginparwidth }{2cm}
\usepackage{todonotes}
\newcommand{\TODO}[2][]
{\todo[size=\scriptsize, color=red, #1]{#2}}

\newcommand{\properties}[1]{\textit{\textbf{Proprietà: }} #1 \vspace{5pt}}
\newcommand{\example}[1]{\textbf{\underline{Esempio:}}  #1}


\begin{document}

\documentclass[a4paper,12 pt]{report}
\usepackage[T1]{fontenc}
\usepackage[utf8]{inputenc}
\usepackage[italian]{babel}
\usepackage{lmodern}
\usepackage{listings}
\usepackage{latexsym}
\usepackage{graphicx}
\usepackage{float}
\usepackage{subcaption}
\usepackage{hyperref}
\usepackage{wrapfig}
\usepackage{fancyhdr}
\usepackage{amsthm}
\usepackage{amsmath}
\usepackage{amssymb}
\usepackage{amsfonts}
\usepackage{cancel}
\usepackage{tcolorbox}


% forza le footnote a stare il più in basso possibile
\usepackage[bottom]{footmisc}

% stile teoremi
\newtheorem{theorem}{Teorema}[section]
\newtheorem{corollary}{Corollario}[theorem]
\newtheorem{lemma}[theorem]{Lemma}
\newtheorem{axch}{Assioma}[section]
\newtheorem{prop}{Proprietà}[section]
\newtheorem{definition}{Definizione}[section]
\newtheorem{limit}{Limitazione}[section]

\renewcommand*{\proofname}{Dimostrazione}

%% STILE LISTINGS

\usepackage{xcolor}

\definecolor{codegreen}{rgb}{0,0.6,0}
\definecolor{codegray}{rgb}{0.5,0.5,0.5}
\definecolor{codepurple}{rgb}{0.58,0,0.82}
\definecolor{backcolour}{rgb}{0.95,0.95,0.92}

 \lstdefinestyle{mystyle}{
     backgroundcolor=\color{backcolour},   
     commentstyle=\color{codegreen},
     keywordstyle=\bfseries\color{black},
     numberstyle=\tiny\color{codegray},
     stringstyle=\color{codepurple},
     basicstyle=\ttfamily\footnotesize,
     breakatwhitespace=false,         
     breaklines=true,                 
     captionpos=b,                    
     keepspaces=true,                 
     numbers=left,                    
     numbersep=5pt,                  
     showspaces=false,                
     showstringspaces=false,
     showtabs=false,                  
     tabsize=2,
     mathescape=true,
     escapeinside={\%*}{*)},
     morekeywords={
        begin,
        end,
        if,
        then,
        else,
        to,
        endif,
        Procedure,
        while,
        do,
        return,
        true,
        false,
        set,
        for,
        repeat,
        until
        foreach},
 }

 \lstset{style=mystyle}

\renewcommand{\lstlistlistingname}{Elenco dei codici}
\renewcommand{\lstlistingname}{Codice}

\newtcolorbox{definition-box}[1]{%
    colback=codepurple!5!white,%
    colframe=codepurple!75!black,%
    fonttitle=\bfseries,%
    title={#1}
}

% mostra le subsubsection nell'indice
\setcounter{tocdepth}{3}
\setcounter{secnumdepth}{3}

% Resetta la numerazione dei chapter quando
% una nuova part viene creata
\makeatletter
\@addtoreset{chapter}{part}
\makeatother

% Rimuove l'indentazione quando si crea un nuovo paragrafo
\setlength{\parindent}{0pt}

% footer
\pagestyle{fancyplain}
% rimuove la riga nell'header
\fancyhf{} % sets both header and footer to nothing
\renewcommand{\headrulewidth}{0pt}
\fancyfoot[L]{\href{https://github.com/Typing-Monkeys/AppuntiUniversita}{Typing Monkeys}}
\fancyfoot[C]{\emoji{gorilla}}
\fancyfoot[R]{\thepage}

% configurazione emoji
\usepackage{fontspec}
\usepackage{emoji}
\setemojifont{NotoColorEmoji.ttf}[Path=/usr/share/fonts/truetype/noto/]

\setlength {\marginparwidth }{2cm}
\usepackage{todonotes}
\newcommand{\TODO}[2][]
{\todo[size=\scriptsize, color=red, #1]{#2}}

\newcommand{\properties}[1]{\textit{\textbf{Proprietà: }} #1 \vspace{5pt}}
\newcommand{\example}[1]{\textbf{\underline{Esempio:}}  #1}


\begin{document}

\documentclass[a4paper,12 pt]{report}
\usepackage[T1]{fontenc}
\usepackage[utf8]{inputenc}
\usepackage[italian]{babel}
\usepackage{lmodern}
\usepackage{listings}
\usepackage{latexsym}
\usepackage{graphicx}
\usepackage{float}
\usepackage{subcaption}
\usepackage{hyperref}
\usepackage{wrapfig}
\usepackage{fancyhdr}
\usepackage{amsthm}
\usepackage{amsmath}
\usepackage{amssymb}
\usepackage{amsfonts}
\usepackage{cancel}
\usepackage{tcolorbox}


% forza le footnote a stare il più in basso possibile
\usepackage[bottom]{footmisc}

% stile teoremi
\newtheorem{theorem}{Teorema}[section]
\newtheorem{corollary}{Corollario}[theorem]
\newtheorem{lemma}[theorem]{Lemma}
\newtheorem{axch}{Assioma}[section]
\newtheorem{prop}{Proprietà}[section]
\newtheorem{definition}{Definizione}[section]
\newtheorem{limit}{Limitazione}[section]

\renewcommand*{\proofname}{Dimostrazione}

%% STILE LISTINGS

\usepackage{xcolor}

\definecolor{codegreen}{rgb}{0,0.6,0}
\definecolor{codegray}{rgb}{0.5,0.5,0.5}
\definecolor{codepurple}{rgb}{0.58,0,0.82}
\definecolor{backcolour}{rgb}{0.95,0.95,0.92}

 \lstdefinestyle{mystyle}{
     backgroundcolor=\color{backcolour},   
     commentstyle=\color{codegreen},
     keywordstyle=\bfseries\color{black},
     numberstyle=\tiny\color{codegray},
     stringstyle=\color{codepurple},
     basicstyle=\ttfamily\footnotesize,
     breakatwhitespace=false,         
     breaklines=true,                 
     captionpos=b,                    
     keepspaces=true,                 
     numbers=left,                    
     numbersep=5pt,                  
     showspaces=false,                
     showstringspaces=false,
     showtabs=false,                  
     tabsize=2,
     mathescape=true,
     escapeinside={\%*}{*)},
     morekeywords={
        begin,
        end,
        if,
        then,
        else,
        to,
        endif,
        Procedure,
        while,
        do,
        return,
        true,
        false,
        set,
        for,
        repeat,
        until
        foreach},
 }

 \lstset{style=mystyle}

\renewcommand{\lstlistlistingname}{Elenco dei codici}
\renewcommand{\lstlistingname}{Codice}

\newtcolorbox{definition-box}[1]{%
    colback=codepurple!5!white,%
    colframe=codepurple!75!black,%
    fonttitle=\bfseries,%
    title={#1}
}

% mostra le subsubsection nell'indice
\setcounter{tocdepth}{3}
\setcounter{secnumdepth}{3}

% Resetta la numerazione dei chapter quando
% una nuova part viene creata
\makeatletter
\@addtoreset{chapter}{part}
\makeatother

% Rimuove l'indentazione quando si crea un nuovo paragrafo
\setlength{\parindent}{0pt}

% footer
\pagestyle{fancyplain}
% rimuove la riga nell'header
\fancyhf{} % sets both header and footer to nothing
\renewcommand{\headrulewidth}{0pt}
\fancyfoot[L]{\href{https://github.com/Typing-Monkeys/AppuntiUniversita}{Typing Monkeys}}
\fancyfoot[C]{\emoji{gorilla}}
\fancyfoot[R]{\thepage}

% configurazione emoji
\usepackage{fontspec}
\usepackage{emoji}
\setemojifont{NotoColorEmoji.ttf}[Path=/usr/share/fonts/truetype/noto/]

\setlength {\marginparwidth }{2cm}
\usepackage{todonotes}
\newcommand{\TODO}[2][]
{\todo[size=\scriptsize, color=red, #1]{#2}}

\newcommand{\properties}[1]{\textit{\textbf{Proprietà: }} #1 \vspace{5pt}}
\newcommand{\example}[1]{\textbf{\underline{Esempio:}}  #1}


\begin{document}

\documentclass[a4paper,12 pt]{report}
\usepackage[T1]{fontenc}
\usepackage[utf8]{inputenc}
\usepackage[italian]{babel}
\usepackage{lmodern}
\usepackage{listings}
\usepackage{latexsym}
\usepackage{graphicx}
\usepackage{float}
\usepackage{subcaption}
\usepackage{hyperref}
\usepackage{wrapfig}
\usepackage{fancyhdr}
\usepackage{amsthm}
\usepackage{amsmath}
\usepackage{amssymb}
\usepackage{amsfonts}
\usepackage{cancel}
\usepackage{tcolorbox}


% forza le footnote a stare il più in basso possibile
\usepackage[bottom]{footmisc}

% stile teoremi
\newtheorem{theorem}{Teorema}[section]
\newtheorem{corollary}{Corollario}[theorem]
\newtheorem{lemma}[theorem]{Lemma}
\newtheorem{axch}{Assioma}[section]
\newtheorem{prop}{Proprietà}[section]
\newtheorem{definition}{Definizione}[section]
\newtheorem{limit}{Limitazione}[section]

\renewcommand*{\proofname}{Dimostrazione}

%% STILE LISTINGS

\usepackage{xcolor}

\definecolor{codegreen}{rgb}{0,0.6,0}
\definecolor{codegray}{rgb}{0.5,0.5,0.5}
\definecolor{codepurple}{rgb}{0.58,0,0.82}
\definecolor{backcolour}{rgb}{0.95,0.95,0.92}

 \lstdefinestyle{mystyle}{
     backgroundcolor=\color{backcolour},   
     commentstyle=\color{codegreen},
     keywordstyle=\bfseries\color{black},
     numberstyle=\tiny\color{codegray},
     stringstyle=\color{codepurple},
     basicstyle=\ttfamily\footnotesize,
     breakatwhitespace=false,         
     breaklines=true,                 
     captionpos=b,                    
     keepspaces=true,                 
     numbers=left,                    
     numbersep=5pt,                  
     showspaces=false,                
     showstringspaces=false,
     showtabs=false,                  
     tabsize=2,
     mathescape=true,
     escapeinside={\%*}{*)},
     morekeywords={
        begin,
        end,
        if,
        then,
        else,
        to,
        endif,
        Procedure,
        while,
        do,
        return,
        true,
        false,
        set,
        for,
        repeat,
        until
        foreach},
 }

 \lstset{style=mystyle}

\renewcommand{\lstlistlistingname}{Elenco dei codici}
\renewcommand{\lstlistingname}{Codice}

\newtcolorbox{definition-box}[1]{%
    colback=codepurple!5!white,%
    colframe=codepurple!75!black,%
    fonttitle=\bfseries,%
    title={#1}
}

% mostra le subsubsection nell'indice
\setcounter{tocdepth}{3}
\setcounter{secnumdepth}{3}

% Resetta la numerazione dei chapter quando
% una nuova part viene creata
\makeatletter
\@addtoreset{chapter}{part}
\makeatother

% Rimuove l'indentazione quando si crea un nuovo paragrafo
\setlength{\parindent}{0pt}

% footer
\pagestyle{fancyplain}
% rimuove la riga nell'header
\fancyhf{} % sets both header and footer to nothing
\renewcommand{\headrulewidth}{0pt}
\fancyfoot[L]{\href{https://github.com/Typing-Monkeys/AppuntiUniversita}{Typing Monkeys}}
\fancyfoot[C]{\emoji{gorilla}}
\fancyfoot[R]{\thepage}

% configurazione emoji
\usepackage{fontspec}
\usepackage{emoji}
\setemojifont{NotoColorEmoji.ttf}[Path=/usr/share/fonts/truetype/noto/]

\setlength {\marginparwidth }{2cm}
\usepackage{todonotes}
\newcommand{\TODO}[2][]
{\todo[size=\scriptsize, color=red, #1]{#2}}

\newcommand{\properties}[1]{\textit{\textbf{Proprietà: }} #1 \vspace{5pt}}
\newcommand{\example}[1]{\textbf{\underline{Esempio:}}  #1}


\begin{document}

\include{frontmatter/main.tex}

%% Aggiungere i capitoli qui sotto
\include{capitoli/introduzione/main}
\include{capitoli/agenti-intelligenti/main}


\end{document}

%% Aggiungere i capitoli qui sotto
\documentclass[a4paper,12 pt]{report}
\usepackage[T1]{fontenc}
\usepackage[utf8]{inputenc}
\usepackage[italian]{babel}
\usepackage{lmodern}
\usepackage{listings}
\usepackage{latexsym}
\usepackage{graphicx}
\usepackage{float}
\usepackage{subcaption}
\usepackage{hyperref}
\usepackage{wrapfig}
\usepackage{fancyhdr}
\usepackage{amsthm}
\usepackage{amsmath}
\usepackage{amssymb}
\usepackage{amsfonts}
\usepackage{cancel}
\usepackage{tcolorbox}


% forza le footnote a stare il più in basso possibile
\usepackage[bottom]{footmisc}

% stile teoremi
\newtheorem{theorem}{Teorema}[section]
\newtheorem{corollary}{Corollario}[theorem]
\newtheorem{lemma}[theorem]{Lemma}
\newtheorem{axch}{Assioma}[section]
\newtheorem{prop}{Proprietà}[section]
\newtheorem{definition}{Definizione}[section]
\newtheorem{limit}{Limitazione}[section]

\renewcommand*{\proofname}{Dimostrazione}

%% STILE LISTINGS

\usepackage{xcolor}

\definecolor{codegreen}{rgb}{0,0.6,0}
\definecolor{codegray}{rgb}{0.5,0.5,0.5}
\definecolor{codepurple}{rgb}{0.58,0,0.82}
\definecolor{backcolour}{rgb}{0.95,0.95,0.92}

 \lstdefinestyle{mystyle}{
     backgroundcolor=\color{backcolour},   
     commentstyle=\color{codegreen},
     keywordstyle=\bfseries\color{black},
     numberstyle=\tiny\color{codegray},
     stringstyle=\color{codepurple},
     basicstyle=\ttfamily\footnotesize,
     breakatwhitespace=false,         
     breaklines=true,                 
     captionpos=b,                    
     keepspaces=true,                 
     numbers=left,                    
     numbersep=5pt,                  
     showspaces=false,                
     showstringspaces=false,
     showtabs=false,                  
     tabsize=2,
     mathescape=true,
     escapeinside={\%*}{*)},
     morekeywords={
        begin,
        end,
        if,
        then,
        else,
        to,
        endif,
        Procedure,
        while,
        do,
        return,
        true,
        false,
        set,
        for,
        repeat,
        until
        foreach},
 }

 \lstset{style=mystyle}

\renewcommand{\lstlistlistingname}{Elenco dei codici}
\renewcommand{\lstlistingname}{Codice}

\newtcolorbox{definition-box}[1]{%
    colback=codepurple!5!white,%
    colframe=codepurple!75!black,%
    fonttitle=\bfseries,%
    title={#1}
}

% mostra le subsubsection nell'indice
\setcounter{tocdepth}{3}
\setcounter{secnumdepth}{3}

% Resetta la numerazione dei chapter quando
% una nuova part viene creata
\makeatletter
\@addtoreset{chapter}{part}
\makeatother

% Rimuove l'indentazione quando si crea un nuovo paragrafo
\setlength{\parindent}{0pt}

% footer
\pagestyle{fancyplain}
% rimuove la riga nell'header
\fancyhf{} % sets both header and footer to nothing
\renewcommand{\headrulewidth}{0pt}
\fancyfoot[L]{\href{https://github.com/Typing-Monkeys/AppuntiUniversita}{Typing Monkeys}}
\fancyfoot[C]{\emoji{gorilla}}
\fancyfoot[R]{\thepage}

% configurazione emoji
\usepackage{fontspec}
\usepackage{emoji}
\setemojifont{NotoColorEmoji.ttf}[Path=/usr/share/fonts/truetype/noto/]

\setlength {\marginparwidth }{2cm}
\usepackage{todonotes}
\newcommand{\TODO}[2][]
{\todo[size=\scriptsize, color=red, #1]{#2}}

\newcommand{\properties}[1]{\textit{\textbf{Proprietà: }} #1 \vspace{5pt}}
\newcommand{\example}[1]{\textbf{\underline{Esempio:}}  #1}


\begin{document}

\include{frontmatter/main.tex}

%% Aggiungere i capitoli qui sotto
\include{capitoli/introduzione/main}
\include{capitoli/agenti-intelligenti/main}


\end{document}
\documentclass[a4paper,12 pt]{report}
\usepackage[T1]{fontenc}
\usepackage[utf8]{inputenc}
\usepackage[italian]{babel}
\usepackage{lmodern}
\usepackage{listings}
\usepackage{latexsym}
\usepackage{graphicx}
\usepackage{float}
\usepackage{subcaption}
\usepackage{hyperref}
\usepackage{wrapfig}
\usepackage{fancyhdr}
\usepackage{amsthm}
\usepackage{amsmath}
\usepackage{amssymb}
\usepackage{amsfonts}
\usepackage{cancel}
\usepackage{tcolorbox}


% forza le footnote a stare il più in basso possibile
\usepackage[bottom]{footmisc}

% stile teoremi
\newtheorem{theorem}{Teorema}[section]
\newtheorem{corollary}{Corollario}[theorem]
\newtheorem{lemma}[theorem]{Lemma}
\newtheorem{axch}{Assioma}[section]
\newtheorem{prop}{Proprietà}[section]
\newtheorem{definition}{Definizione}[section]
\newtheorem{limit}{Limitazione}[section]

\renewcommand*{\proofname}{Dimostrazione}

%% STILE LISTINGS

\usepackage{xcolor}

\definecolor{codegreen}{rgb}{0,0.6,0}
\definecolor{codegray}{rgb}{0.5,0.5,0.5}
\definecolor{codepurple}{rgb}{0.58,0,0.82}
\definecolor{backcolour}{rgb}{0.95,0.95,0.92}

 \lstdefinestyle{mystyle}{
     backgroundcolor=\color{backcolour},   
     commentstyle=\color{codegreen},
     keywordstyle=\bfseries\color{black},
     numberstyle=\tiny\color{codegray},
     stringstyle=\color{codepurple},
     basicstyle=\ttfamily\footnotesize,
     breakatwhitespace=false,         
     breaklines=true,                 
     captionpos=b,                    
     keepspaces=true,                 
     numbers=left,                    
     numbersep=5pt,                  
     showspaces=false,                
     showstringspaces=false,
     showtabs=false,                  
     tabsize=2,
     mathescape=true,
     escapeinside={\%*}{*)},
     morekeywords={
        begin,
        end,
        if,
        then,
        else,
        to,
        endif,
        Procedure,
        while,
        do,
        return,
        true,
        false,
        set,
        for,
        repeat,
        until
        foreach},
 }

 \lstset{style=mystyle}

\renewcommand{\lstlistlistingname}{Elenco dei codici}
\renewcommand{\lstlistingname}{Codice}

\newtcolorbox{definition-box}[1]{%
    colback=codepurple!5!white,%
    colframe=codepurple!75!black,%
    fonttitle=\bfseries,%
    title={#1}
}

% mostra le subsubsection nell'indice
\setcounter{tocdepth}{3}
\setcounter{secnumdepth}{3}

% Resetta la numerazione dei chapter quando
% una nuova part viene creata
\makeatletter
\@addtoreset{chapter}{part}
\makeatother

% Rimuove l'indentazione quando si crea un nuovo paragrafo
\setlength{\parindent}{0pt}

% footer
\pagestyle{fancyplain}
% rimuove la riga nell'header
\fancyhf{} % sets both header and footer to nothing
\renewcommand{\headrulewidth}{0pt}
\fancyfoot[L]{\href{https://github.com/Typing-Monkeys/AppuntiUniversita}{Typing Monkeys}}
\fancyfoot[C]{\emoji{gorilla}}
\fancyfoot[R]{\thepage}

% configurazione emoji
\usepackage{fontspec}
\usepackage{emoji}
\setemojifont{NotoColorEmoji.ttf}[Path=/usr/share/fonts/truetype/noto/]

\setlength {\marginparwidth }{2cm}
\usepackage{todonotes}
\newcommand{\TODO}[2][]
{\todo[size=\scriptsize, color=red, #1]{#2}}

\newcommand{\properties}[1]{\textit{\textbf{Proprietà: }} #1 \vspace{5pt}}
\newcommand{\example}[1]{\textbf{\underline{Esempio:}}  #1}


\begin{document}

\include{frontmatter/main.tex}

%% Aggiungere i capitoli qui sotto
\include{capitoli/introduzione/main}
\include{capitoli/agenti-intelligenti/main}


\end{document}


\end{document}

%% Aggiungere i capitoli qui sotto
\documentclass[a4paper,12 pt]{report}
\usepackage[T1]{fontenc}
\usepackage[utf8]{inputenc}
\usepackage[italian]{babel}
\usepackage{lmodern}
\usepackage{listings}
\usepackage{latexsym}
\usepackage{graphicx}
\usepackage{float}
\usepackage{subcaption}
\usepackage{hyperref}
\usepackage{wrapfig}
\usepackage{fancyhdr}
\usepackage{amsthm}
\usepackage{amsmath}
\usepackage{amssymb}
\usepackage{amsfonts}
\usepackage{cancel}
\usepackage{tcolorbox}


% forza le footnote a stare il più in basso possibile
\usepackage[bottom]{footmisc}

% stile teoremi
\newtheorem{theorem}{Teorema}[section]
\newtheorem{corollary}{Corollario}[theorem]
\newtheorem{lemma}[theorem]{Lemma}
\newtheorem{axch}{Assioma}[section]
\newtheorem{prop}{Proprietà}[section]
\newtheorem{definition}{Definizione}[section]
\newtheorem{limit}{Limitazione}[section]

\renewcommand*{\proofname}{Dimostrazione}

%% STILE LISTINGS

\usepackage{xcolor}

\definecolor{codegreen}{rgb}{0,0.6,0}
\definecolor{codegray}{rgb}{0.5,0.5,0.5}
\definecolor{codepurple}{rgb}{0.58,0,0.82}
\definecolor{backcolour}{rgb}{0.95,0.95,0.92}

 \lstdefinestyle{mystyle}{
     backgroundcolor=\color{backcolour},   
     commentstyle=\color{codegreen},
     keywordstyle=\bfseries\color{black},
     numberstyle=\tiny\color{codegray},
     stringstyle=\color{codepurple},
     basicstyle=\ttfamily\footnotesize,
     breakatwhitespace=false,         
     breaklines=true,                 
     captionpos=b,                    
     keepspaces=true,                 
     numbers=left,                    
     numbersep=5pt,                  
     showspaces=false,                
     showstringspaces=false,
     showtabs=false,                  
     tabsize=2,
     mathescape=true,
     escapeinside={\%*}{*)},
     morekeywords={
        begin,
        end,
        if,
        then,
        else,
        to,
        endif,
        Procedure,
        while,
        do,
        return,
        true,
        false,
        set,
        for,
        repeat,
        until
        foreach},
 }

 \lstset{style=mystyle}

\renewcommand{\lstlistlistingname}{Elenco dei codici}
\renewcommand{\lstlistingname}{Codice}

\newtcolorbox{definition-box}[1]{%
    colback=codepurple!5!white,%
    colframe=codepurple!75!black,%
    fonttitle=\bfseries,%
    title={#1}
}

% mostra le subsubsection nell'indice
\setcounter{tocdepth}{3}
\setcounter{secnumdepth}{3}

% Resetta la numerazione dei chapter quando
% una nuova part viene creata
\makeatletter
\@addtoreset{chapter}{part}
\makeatother

% Rimuove l'indentazione quando si crea un nuovo paragrafo
\setlength{\parindent}{0pt}

% footer
\pagestyle{fancyplain}
% rimuove la riga nell'header
\fancyhf{} % sets both header and footer to nothing
\renewcommand{\headrulewidth}{0pt}
\fancyfoot[L]{\href{https://github.com/Typing-Monkeys/AppuntiUniversita}{Typing Monkeys}}
\fancyfoot[C]{\emoji{gorilla}}
\fancyfoot[R]{\thepage}

% configurazione emoji
\usepackage{fontspec}
\usepackage{emoji}
\setemojifont{NotoColorEmoji.ttf}[Path=/usr/share/fonts/truetype/noto/]

\setlength {\marginparwidth }{2cm}
\usepackage{todonotes}
\newcommand{\TODO}[2][]
{\todo[size=\scriptsize, color=red, #1]{#2}}

\newcommand{\properties}[1]{\textit{\textbf{Proprietà: }} #1 \vspace{5pt}}
\newcommand{\example}[1]{\textbf{\underline{Esempio:}}  #1}


\begin{document}

\documentclass[a4paper,12 pt]{report}
\usepackage[T1]{fontenc}
\usepackage[utf8]{inputenc}
\usepackage[italian]{babel}
\usepackage{lmodern}
\usepackage{listings}
\usepackage{latexsym}
\usepackage{graphicx}
\usepackage{float}
\usepackage{subcaption}
\usepackage{hyperref}
\usepackage{wrapfig}
\usepackage{fancyhdr}
\usepackage{amsthm}
\usepackage{amsmath}
\usepackage{amssymb}
\usepackage{amsfonts}
\usepackage{cancel}
\usepackage{tcolorbox}


% forza le footnote a stare il più in basso possibile
\usepackage[bottom]{footmisc}

% stile teoremi
\newtheorem{theorem}{Teorema}[section]
\newtheorem{corollary}{Corollario}[theorem]
\newtheorem{lemma}[theorem]{Lemma}
\newtheorem{axch}{Assioma}[section]
\newtheorem{prop}{Proprietà}[section]
\newtheorem{definition}{Definizione}[section]
\newtheorem{limit}{Limitazione}[section]

\renewcommand*{\proofname}{Dimostrazione}

%% STILE LISTINGS

\usepackage{xcolor}

\definecolor{codegreen}{rgb}{0,0.6,0}
\definecolor{codegray}{rgb}{0.5,0.5,0.5}
\definecolor{codepurple}{rgb}{0.58,0,0.82}
\definecolor{backcolour}{rgb}{0.95,0.95,0.92}

 \lstdefinestyle{mystyle}{
     backgroundcolor=\color{backcolour},   
     commentstyle=\color{codegreen},
     keywordstyle=\bfseries\color{black},
     numberstyle=\tiny\color{codegray},
     stringstyle=\color{codepurple},
     basicstyle=\ttfamily\footnotesize,
     breakatwhitespace=false,         
     breaklines=true,                 
     captionpos=b,                    
     keepspaces=true,                 
     numbers=left,                    
     numbersep=5pt,                  
     showspaces=false,                
     showstringspaces=false,
     showtabs=false,                  
     tabsize=2,
     mathescape=true,
     escapeinside={\%*}{*)},
     morekeywords={
        begin,
        end,
        if,
        then,
        else,
        to,
        endif,
        Procedure,
        while,
        do,
        return,
        true,
        false,
        set,
        for,
        repeat,
        until
        foreach},
 }

 \lstset{style=mystyle}

\renewcommand{\lstlistlistingname}{Elenco dei codici}
\renewcommand{\lstlistingname}{Codice}

\newtcolorbox{definition-box}[1]{%
    colback=codepurple!5!white,%
    colframe=codepurple!75!black,%
    fonttitle=\bfseries,%
    title={#1}
}

% mostra le subsubsection nell'indice
\setcounter{tocdepth}{3}
\setcounter{secnumdepth}{3}

% Resetta la numerazione dei chapter quando
% una nuova part viene creata
\makeatletter
\@addtoreset{chapter}{part}
\makeatother

% Rimuove l'indentazione quando si crea un nuovo paragrafo
\setlength{\parindent}{0pt}

% footer
\pagestyle{fancyplain}
% rimuove la riga nell'header
\fancyhf{} % sets both header and footer to nothing
\renewcommand{\headrulewidth}{0pt}
\fancyfoot[L]{\href{https://github.com/Typing-Monkeys/AppuntiUniversita}{Typing Monkeys}}
\fancyfoot[C]{\emoji{gorilla}}
\fancyfoot[R]{\thepage}

% configurazione emoji
\usepackage{fontspec}
\usepackage{emoji}
\setemojifont{NotoColorEmoji.ttf}[Path=/usr/share/fonts/truetype/noto/]

\setlength {\marginparwidth }{2cm}
\usepackage{todonotes}
\newcommand{\TODO}[2][]
{\todo[size=\scriptsize, color=red, #1]{#2}}

\newcommand{\properties}[1]{\textit{\textbf{Proprietà: }} #1 \vspace{5pt}}
\newcommand{\example}[1]{\textbf{\underline{Esempio:}}  #1}


\begin{document}

\include{frontmatter/main.tex}

%% Aggiungere i capitoli qui sotto
\include{capitoli/introduzione/main}
\include{capitoli/agenti-intelligenti/main}


\end{document}

%% Aggiungere i capitoli qui sotto
\documentclass[a4paper,12 pt]{report}
\usepackage[T1]{fontenc}
\usepackage[utf8]{inputenc}
\usepackage[italian]{babel}
\usepackage{lmodern}
\usepackage{listings}
\usepackage{latexsym}
\usepackage{graphicx}
\usepackage{float}
\usepackage{subcaption}
\usepackage{hyperref}
\usepackage{wrapfig}
\usepackage{fancyhdr}
\usepackage{amsthm}
\usepackage{amsmath}
\usepackage{amssymb}
\usepackage{amsfonts}
\usepackage{cancel}
\usepackage{tcolorbox}


% forza le footnote a stare il più in basso possibile
\usepackage[bottom]{footmisc}

% stile teoremi
\newtheorem{theorem}{Teorema}[section]
\newtheorem{corollary}{Corollario}[theorem]
\newtheorem{lemma}[theorem]{Lemma}
\newtheorem{axch}{Assioma}[section]
\newtheorem{prop}{Proprietà}[section]
\newtheorem{definition}{Definizione}[section]
\newtheorem{limit}{Limitazione}[section]

\renewcommand*{\proofname}{Dimostrazione}

%% STILE LISTINGS

\usepackage{xcolor}

\definecolor{codegreen}{rgb}{0,0.6,0}
\definecolor{codegray}{rgb}{0.5,0.5,0.5}
\definecolor{codepurple}{rgb}{0.58,0,0.82}
\definecolor{backcolour}{rgb}{0.95,0.95,0.92}

 \lstdefinestyle{mystyle}{
     backgroundcolor=\color{backcolour},   
     commentstyle=\color{codegreen},
     keywordstyle=\bfseries\color{black},
     numberstyle=\tiny\color{codegray},
     stringstyle=\color{codepurple},
     basicstyle=\ttfamily\footnotesize,
     breakatwhitespace=false,         
     breaklines=true,                 
     captionpos=b,                    
     keepspaces=true,                 
     numbers=left,                    
     numbersep=5pt,                  
     showspaces=false,                
     showstringspaces=false,
     showtabs=false,                  
     tabsize=2,
     mathescape=true,
     escapeinside={\%*}{*)},
     morekeywords={
        begin,
        end,
        if,
        then,
        else,
        to,
        endif,
        Procedure,
        while,
        do,
        return,
        true,
        false,
        set,
        for,
        repeat,
        until
        foreach},
 }

 \lstset{style=mystyle}

\renewcommand{\lstlistlistingname}{Elenco dei codici}
\renewcommand{\lstlistingname}{Codice}

\newtcolorbox{definition-box}[1]{%
    colback=codepurple!5!white,%
    colframe=codepurple!75!black,%
    fonttitle=\bfseries,%
    title={#1}
}

% mostra le subsubsection nell'indice
\setcounter{tocdepth}{3}
\setcounter{secnumdepth}{3}

% Resetta la numerazione dei chapter quando
% una nuova part viene creata
\makeatletter
\@addtoreset{chapter}{part}
\makeatother

% Rimuove l'indentazione quando si crea un nuovo paragrafo
\setlength{\parindent}{0pt}

% footer
\pagestyle{fancyplain}
% rimuove la riga nell'header
\fancyhf{} % sets both header and footer to nothing
\renewcommand{\headrulewidth}{0pt}
\fancyfoot[L]{\href{https://github.com/Typing-Monkeys/AppuntiUniversita}{Typing Monkeys}}
\fancyfoot[C]{\emoji{gorilla}}
\fancyfoot[R]{\thepage}

% configurazione emoji
\usepackage{fontspec}
\usepackage{emoji}
\setemojifont{NotoColorEmoji.ttf}[Path=/usr/share/fonts/truetype/noto/]

\setlength {\marginparwidth }{2cm}
\usepackage{todonotes}
\newcommand{\TODO}[2][]
{\todo[size=\scriptsize, color=red, #1]{#2}}

\newcommand{\properties}[1]{\textit{\textbf{Proprietà: }} #1 \vspace{5pt}}
\newcommand{\example}[1]{\textbf{\underline{Esempio:}}  #1}


\begin{document}

\include{frontmatter/main.tex}

%% Aggiungere i capitoli qui sotto
\include{capitoli/introduzione/main}
\include{capitoli/agenti-intelligenti/main}


\end{document}
\documentclass[a4paper,12 pt]{report}
\usepackage[T1]{fontenc}
\usepackage[utf8]{inputenc}
\usepackage[italian]{babel}
\usepackage{lmodern}
\usepackage{listings}
\usepackage{latexsym}
\usepackage{graphicx}
\usepackage{float}
\usepackage{subcaption}
\usepackage{hyperref}
\usepackage{wrapfig}
\usepackage{fancyhdr}
\usepackage{amsthm}
\usepackage{amsmath}
\usepackage{amssymb}
\usepackage{amsfonts}
\usepackage{cancel}
\usepackage{tcolorbox}


% forza le footnote a stare il più in basso possibile
\usepackage[bottom]{footmisc}

% stile teoremi
\newtheorem{theorem}{Teorema}[section]
\newtheorem{corollary}{Corollario}[theorem]
\newtheorem{lemma}[theorem]{Lemma}
\newtheorem{axch}{Assioma}[section]
\newtheorem{prop}{Proprietà}[section]
\newtheorem{definition}{Definizione}[section]
\newtheorem{limit}{Limitazione}[section]

\renewcommand*{\proofname}{Dimostrazione}

%% STILE LISTINGS

\usepackage{xcolor}

\definecolor{codegreen}{rgb}{0,0.6,0}
\definecolor{codegray}{rgb}{0.5,0.5,0.5}
\definecolor{codepurple}{rgb}{0.58,0,0.82}
\definecolor{backcolour}{rgb}{0.95,0.95,0.92}

 \lstdefinestyle{mystyle}{
     backgroundcolor=\color{backcolour},   
     commentstyle=\color{codegreen},
     keywordstyle=\bfseries\color{black},
     numberstyle=\tiny\color{codegray},
     stringstyle=\color{codepurple},
     basicstyle=\ttfamily\footnotesize,
     breakatwhitespace=false,         
     breaklines=true,                 
     captionpos=b,                    
     keepspaces=true,                 
     numbers=left,                    
     numbersep=5pt,                  
     showspaces=false,                
     showstringspaces=false,
     showtabs=false,                  
     tabsize=2,
     mathescape=true,
     escapeinside={\%*}{*)},
     morekeywords={
        begin,
        end,
        if,
        then,
        else,
        to,
        endif,
        Procedure,
        while,
        do,
        return,
        true,
        false,
        set,
        for,
        repeat,
        until
        foreach},
 }

 \lstset{style=mystyle}

\renewcommand{\lstlistlistingname}{Elenco dei codici}
\renewcommand{\lstlistingname}{Codice}

\newtcolorbox{definition-box}[1]{%
    colback=codepurple!5!white,%
    colframe=codepurple!75!black,%
    fonttitle=\bfseries,%
    title={#1}
}

% mostra le subsubsection nell'indice
\setcounter{tocdepth}{3}
\setcounter{secnumdepth}{3}

% Resetta la numerazione dei chapter quando
% una nuova part viene creata
\makeatletter
\@addtoreset{chapter}{part}
\makeatother

% Rimuove l'indentazione quando si crea un nuovo paragrafo
\setlength{\parindent}{0pt}

% footer
\pagestyle{fancyplain}
% rimuove la riga nell'header
\fancyhf{} % sets both header and footer to nothing
\renewcommand{\headrulewidth}{0pt}
\fancyfoot[L]{\href{https://github.com/Typing-Monkeys/AppuntiUniversita}{Typing Monkeys}}
\fancyfoot[C]{\emoji{gorilla}}
\fancyfoot[R]{\thepage}

% configurazione emoji
\usepackage{fontspec}
\usepackage{emoji}
\setemojifont{NotoColorEmoji.ttf}[Path=/usr/share/fonts/truetype/noto/]

\setlength {\marginparwidth }{2cm}
\usepackage{todonotes}
\newcommand{\TODO}[2][]
{\todo[size=\scriptsize, color=red, #1]{#2}}

\newcommand{\properties}[1]{\textit{\textbf{Proprietà: }} #1 \vspace{5pt}}
\newcommand{\example}[1]{\textbf{\underline{Esempio:}}  #1}


\begin{document}

\include{frontmatter/main.tex}

%% Aggiungere i capitoli qui sotto
\include{capitoli/introduzione/main}
\include{capitoli/agenti-intelligenti/main}


\end{document}


\end{document}
\documentclass[a4paper,12 pt]{report}
\usepackage[T1]{fontenc}
\usepackage[utf8]{inputenc}
\usepackage[italian]{babel}
\usepackage{lmodern}
\usepackage{listings}
\usepackage{latexsym}
\usepackage{graphicx}
\usepackage{float}
\usepackage{subcaption}
\usepackage{hyperref}
\usepackage{wrapfig}
\usepackage{fancyhdr}
\usepackage{amsthm}
\usepackage{amsmath}
\usepackage{amssymb}
\usepackage{amsfonts}
\usepackage{cancel}
\usepackage{tcolorbox}


% forza le footnote a stare il più in basso possibile
\usepackage[bottom]{footmisc}

% stile teoremi
\newtheorem{theorem}{Teorema}[section]
\newtheorem{corollary}{Corollario}[theorem]
\newtheorem{lemma}[theorem]{Lemma}
\newtheorem{axch}{Assioma}[section]
\newtheorem{prop}{Proprietà}[section]
\newtheorem{definition}{Definizione}[section]
\newtheorem{limit}{Limitazione}[section]

\renewcommand*{\proofname}{Dimostrazione}

%% STILE LISTINGS

\usepackage{xcolor}

\definecolor{codegreen}{rgb}{0,0.6,0}
\definecolor{codegray}{rgb}{0.5,0.5,0.5}
\definecolor{codepurple}{rgb}{0.58,0,0.82}
\definecolor{backcolour}{rgb}{0.95,0.95,0.92}

 \lstdefinestyle{mystyle}{
     backgroundcolor=\color{backcolour},   
     commentstyle=\color{codegreen},
     keywordstyle=\bfseries\color{black},
     numberstyle=\tiny\color{codegray},
     stringstyle=\color{codepurple},
     basicstyle=\ttfamily\footnotesize,
     breakatwhitespace=false,         
     breaklines=true,                 
     captionpos=b,                    
     keepspaces=true,                 
     numbers=left,                    
     numbersep=5pt,                  
     showspaces=false,                
     showstringspaces=false,
     showtabs=false,                  
     tabsize=2,
     mathescape=true,
     escapeinside={\%*}{*)},
     morekeywords={
        begin,
        end,
        if,
        then,
        else,
        to,
        endif,
        Procedure,
        while,
        do,
        return,
        true,
        false,
        set,
        for,
        repeat,
        until
        foreach},
 }

 \lstset{style=mystyle}

\renewcommand{\lstlistlistingname}{Elenco dei codici}
\renewcommand{\lstlistingname}{Codice}

\newtcolorbox{definition-box}[1]{%
    colback=codepurple!5!white,%
    colframe=codepurple!75!black,%
    fonttitle=\bfseries,%
    title={#1}
}

% mostra le subsubsection nell'indice
\setcounter{tocdepth}{3}
\setcounter{secnumdepth}{3}

% Resetta la numerazione dei chapter quando
% una nuova part viene creata
\makeatletter
\@addtoreset{chapter}{part}
\makeatother

% Rimuove l'indentazione quando si crea un nuovo paragrafo
\setlength{\parindent}{0pt}

% footer
\pagestyle{fancyplain}
% rimuove la riga nell'header
\fancyhf{} % sets both header and footer to nothing
\renewcommand{\headrulewidth}{0pt}
\fancyfoot[L]{\href{https://github.com/Typing-Monkeys/AppuntiUniversita}{Typing Monkeys}}
\fancyfoot[C]{\emoji{gorilla}}
\fancyfoot[R]{\thepage}

% configurazione emoji
\usepackage{fontspec}
\usepackage{emoji}
\setemojifont{NotoColorEmoji.ttf}[Path=/usr/share/fonts/truetype/noto/]

\setlength {\marginparwidth }{2cm}
\usepackage{todonotes}
\newcommand{\TODO}[2][]
{\todo[size=\scriptsize, color=red, #1]{#2}}

\newcommand{\properties}[1]{\textit{\textbf{Proprietà: }} #1 \vspace{5pt}}
\newcommand{\example}[1]{\textbf{\underline{Esempio:}}  #1}


\begin{document}

\documentclass[a4paper,12 pt]{report}
\usepackage[T1]{fontenc}
\usepackage[utf8]{inputenc}
\usepackage[italian]{babel}
\usepackage{lmodern}
\usepackage{listings}
\usepackage{latexsym}
\usepackage{graphicx}
\usepackage{float}
\usepackage{subcaption}
\usepackage{hyperref}
\usepackage{wrapfig}
\usepackage{fancyhdr}
\usepackage{amsthm}
\usepackage{amsmath}
\usepackage{amssymb}
\usepackage{amsfonts}
\usepackage{cancel}
\usepackage{tcolorbox}


% forza le footnote a stare il più in basso possibile
\usepackage[bottom]{footmisc}

% stile teoremi
\newtheorem{theorem}{Teorema}[section]
\newtheorem{corollary}{Corollario}[theorem]
\newtheorem{lemma}[theorem]{Lemma}
\newtheorem{axch}{Assioma}[section]
\newtheorem{prop}{Proprietà}[section]
\newtheorem{definition}{Definizione}[section]
\newtheorem{limit}{Limitazione}[section]

\renewcommand*{\proofname}{Dimostrazione}

%% STILE LISTINGS

\usepackage{xcolor}

\definecolor{codegreen}{rgb}{0,0.6,0}
\definecolor{codegray}{rgb}{0.5,0.5,0.5}
\definecolor{codepurple}{rgb}{0.58,0,0.82}
\definecolor{backcolour}{rgb}{0.95,0.95,0.92}

 \lstdefinestyle{mystyle}{
     backgroundcolor=\color{backcolour},   
     commentstyle=\color{codegreen},
     keywordstyle=\bfseries\color{black},
     numberstyle=\tiny\color{codegray},
     stringstyle=\color{codepurple},
     basicstyle=\ttfamily\footnotesize,
     breakatwhitespace=false,         
     breaklines=true,                 
     captionpos=b,                    
     keepspaces=true,                 
     numbers=left,                    
     numbersep=5pt,                  
     showspaces=false,                
     showstringspaces=false,
     showtabs=false,                  
     tabsize=2,
     mathescape=true,
     escapeinside={\%*}{*)},
     morekeywords={
        begin,
        end,
        if,
        then,
        else,
        to,
        endif,
        Procedure,
        while,
        do,
        return,
        true,
        false,
        set,
        for,
        repeat,
        until
        foreach},
 }

 \lstset{style=mystyle}

\renewcommand{\lstlistlistingname}{Elenco dei codici}
\renewcommand{\lstlistingname}{Codice}

\newtcolorbox{definition-box}[1]{%
    colback=codepurple!5!white,%
    colframe=codepurple!75!black,%
    fonttitle=\bfseries,%
    title={#1}
}

% mostra le subsubsection nell'indice
\setcounter{tocdepth}{3}
\setcounter{secnumdepth}{3}

% Resetta la numerazione dei chapter quando
% una nuova part viene creata
\makeatletter
\@addtoreset{chapter}{part}
\makeatother

% Rimuove l'indentazione quando si crea un nuovo paragrafo
\setlength{\parindent}{0pt}

% footer
\pagestyle{fancyplain}
% rimuove la riga nell'header
\fancyhf{} % sets both header and footer to nothing
\renewcommand{\headrulewidth}{0pt}
\fancyfoot[L]{\href{https://github.com/Typing-Monkeys/AppuntiUniversita}{Typing Monkeys}}
\fancyfoot[C]{\emoji{gorilla}}
\fancyfoot[R]{\thepage}

% configurazione emoji
\usepackage{fontspec}
\usepackage{emoji}
\setemojifont{NotoColorEmoji.ttf}[Path=/usr/share/fonts/truetype/noto/]

\setlength {\marginparwidth }{2cm}
\usepackage{todonotes}
\newcommand{\TODO}[2][]
{\todo[size=\scriptsize, color=red, #1]{#2}}

\newcommand{\properties}[1]{\textit{\textbf{Proprietà: }} #1 \vspace{5pt}}
\newcommand{\example}[1]{\textbf{\underline{Esempio:}}  #1}


\begin{document}

\include{frontmatter/main.tex}

%% Aggiungere i capitoli qui sotto
\include{capitoli/introduzione/main}
\include{capitoli/agenti-intelligenti/main}


\end{document}

%% Aggiungere i capitoli qui sotto
\documentclass[a4paper,12 pt]{report}
\usepackage[T1]{fontenc}
\usepackage[utf8]{inputenc}
\usepackage[italian]{babel}
\usepackage{lmodern}
\usepackage{listings}
\usepackage{latexsym}
\usepackage{graphicx}
\usepackage{float}
\usepackage{subcaption}
\usepackage{hyperref}
\usepackage{wrapfig}
\usepackage{fancyhdr}
\usepackage{amsthm}
\usepackage{amsmath}
\usepackage{amssymb}
\usepackage{amsfonts}
\usepackage{cancel}
\usepackage{tcolorbox}


% forza le footnote a stare il più in basso possibile
\usepackage[bottom]{footmisc}

% stile teoremi
\newtheorem{theorem}{Teorema}[section]
\newtheorem{corollary}{Corollario}[theorem]
\newtheorem{lemma}[theorem]{Lemma}
\newtheorem{axch}{Assioma}[section]
\newtheorem{prop}{Proprietà}[section]
\newtheorem{definition}{Definizione}[section]
\newtheorem{limit}{Limitazione}[section]

\renewcommand*{\proofname}{Dimostrazione}

%% STILE LISTINGS

\usepackage{xcolor}

\definecolor{codegreen}{rgb}{0,0.6,0}
\definecolor{codegray}{rgb}{0.5,0.5,0.5}
\definecolor{codepurple}{rgb}{0.58,0,0.82}
\definecolor{backcolour}{rgb}{0.95,0.95,0.92}

 \lstdefinestyle{mystyle}{
     backgroundcolor=\color{backcolour},   
     commentstyle=\color{codegreen},
     keywordstyle=\bfseries\color{black},
     numberstyle=\tiny\color{codegray},
     stringstyle=\color{codepurple},
     basicstyle=\ttfamily\footnotesize,
     breakatwhitespace=false,         
     breaklines=true,                 
     captionpos=b,                    
     keepspaces=true,                 
     numbers=left,                    
     numbersep=5pt,                  
     showspaces=false,                
     showstringspaces=false,
     showtabs=false,                  
     tabsize=2,
     mathescape=true,
     escapeinside={\%*}{*)},
     morekeywords={
        begin,
        end,
        if,
        then,
        else,
        to,
        endif,
        Procedure,
        while,
        do,
        return,
        true,
        false,
        set,
        for,
        repeat,
        until
        foreach},
 }

 \lstset{style=mystyle}

\renewcommand{\lstlistlistingname}{Elenco dei codici}
\renewcommand{\lstlistingname}{Codice}

\newtcolorbox{definition-box}[1]{%
    colback=codepurple!5!white,%
    colframe=codepurple!75!black,%
    fonttitle=\bfseries,%
    title={#1}
}

% mostra le subsubsection nell'indice
\setcounter{tocdepth}{3}
\setcounter{secnumdepth}{3}

% Resetta la numerazione dei chapter quando
% una nuova part viene creata
\makeatletter
\@addtoreset{chapter}{part}
\makeatother

% Rimuove l'indentazione quando si crea un nuovo paragrafo
\setlength{\parindent}{0pt}

% footer
\pagestyle{fancyplain}
% rimuove la riga nell'header
\fancyhf{} % sets both header and footer to nothing
\renewcommand{\headrulewidth}{0pt}
\fancyfoot[L]{\href{https://github.com/Typing-Monkeys/AppuntiUniversita}{Typing Monkeys}}
\fancyfoot[C]{\emoji{gorilla}}
\fancyfoot[R]{\thepage}

% configurazione emoji
\usepackage{fontspec}
\usepackage{emoji}
\setemojifont{NotoColorEmoji.ttf}[Path=/usr/share/fonts/truetype/noto/]

\setlength {\marginparwidth }{2cm}
\usepackage{todonotes}
\newcommand{\TODO}[2][]
{\todo[size=\scriptsize, color=red, #1]{#2}}

\newcommand{\properties}[1]{\textit{\textbf{Proprietà: }} #1 \vspace{5pt}}
\newcommand{\example}[1]{\textbf{\underline{Esempio:}}  #1}


\begin{document}

\include{frontmatter/main.tex}

%% Aggiungere i capitoli qui sotto
\include{capitoli/introduzione/main}
\include{capitoli/agenti-intelligenti/main}


\end{document}
\documentclass[a4paper,12 pt]{report}
\usepackage[T1]{fontenc}
\usepackage[utf8]{inputenc}
\usepackage[italian]{babel}
\usepackage{lmodern}
\usepackage{listings}
\usepackage{latexsym}
\usepackage{graphicx}
\usepackage{float}
\usepackage{subcaption}
\usepackage{hyperref}
\usepackage{wrapfig}
\usepackage{fancyhdr}
\usepackage{amsthm}
\usepackage{amsmath}
\usepackage{amssymb}
\usepackage{amsfonts}
\usepackage{cancel}
\usepackage{tcolorbox}


% forza le footnote a stare il più in basso possibile
\usepackage[bottom]{footmisc}

% stile teoremi
\newtheorem{theorem}{Teorema}[section]
\newtheorem{corollary}{Corollario}[theorem]
\newtheorem{lemma}[theorem]{Lemma}
\newtheorem{axch}{Assioma}[section]
\newtheorem{prop}{Proprietà}[section]
\newtheorem{definition}{Definizione}[section]
\newtheorem{limit}{Limitazione}[section]

\renewcommand*{\proofname}{Dimostrazione}

%% STILE LISTINGS

\usepackage{xcolor}

\definecolor{codegreen}{rgb}{0,0.6,0}
\definecolor{codegray}{rgb}{0.5,0.5,0.5}
\definecolor{codepurple}{rgb}{0.58,0,0.82}
\definecolor{backcolour}{rgb}{0.95,0.95,0.92}

 \lstdefinestyle{mystyle}{
     backgroundcolor=\color{backcolour},   
     commentstyle=\color{codegreen},
     keywordstyle=\bfseries\color{black},
     numberstyle=\tiny\color{codegray},
     stringstyle=\color{codepurple},
     basicstyle=\ttfamily\footnotesize,
     breakatwhitespace=false,         
     breaklines=true,                 
     captionpos=b,                    
     keepspaces=true,                 
     numbers=left,                    
     numbersep=5pt,                  
     showspaces=false,                
     showstringspaces=false,
     showtabs=false,                  
     tabsize=2,
     mathescape=true,
     escapeinside={\%*}{*)},
     morekeywords={
        begin,
        end,
        if,
        then,
        else,
        to,
        endif,
        Procedure,
        while,
        do,
        return,
        true,
        false,
        set,
        for,
        repeat,
        until
        foreach},
 }

 \lstset{style=mystyle}

\renewcommand{\lstlistlistingname}{Elenco dei codici}
\renewcommand{\lstlistingname}{Codice}

\newtcolorbox{definition-box}[1]{%
    colback=codepurple!5!white,%
    colframe=codepurple!75!black,%
    fonttitle=\bfseries,%
    title={#1}
}

% mostra le subsubsection nell'indice
\setcounter{tocdepth}{3}
\setcounter{secnumdepth}{3}

% Resetta la numerazione dei chapter quando
% una nuova part viene creata
\makeatletter
\@addtoreset{chapter}{part}
\makeatother

% Rimuove l'indentazione quando si crea un nuovo paragrafo
\setlength{\parindent}{0pt}

% footer
\pagestyle{fancyplain}
% rimuove la riga nell'header
\fancyhf{} % sets both header and footer to nothing
\renewcommand{\headrulewidth}{0pt}
\fancyfoot[L]{\href{https://github.com/Typing-Monkeys/AppuntiUniversita}{Typing Monkeys}}
\fancyfoot[C]{\emoji{gorilla}}
\fancyfoot[R]{\thepage}

% configurazione emoji
\usepackage{fontspec}
\usepackage{emoji}
\setemojifont{NotoColorEmoji.ttf}[Path=/usr/share/fonts/truetype/noto/]

\setlength {\marginparwidth }{2cm}
\usepackage{todonotes}
\newcommand{\TODO}[2][]
{\todo[size=\scriptsize, color=red, #1]{#2}}

\newcommand{\properties}[1]{\textit{\textbf{Proprietà: }} #1 \vspace{5pt}}
\newcommand{\example}[1]{\textbf{\underline{Esempio:}}  #1}


\begin{document}

\include{frontmatter/main.tex}

%% Aggiungere i capitoli qui sotto
\include{capitoli/introduzione/main}
\include{capitoli/agenti-intelligenti/main}


\end{document}


\end{document}


\end{document}

%% Aggiungere i capitoli qui sotto
\documentclass[a4paper,12 pt]{report}
\usepackage[T1]{fontenc}
\usepackage[utf8]{inputenc}
\usepackage[italian]{babel}
\usepackage{lmodern}
\usepackage{listings}
\usepackage{latexsym}
\usepackage{graphicx}
\usepackage{float}
\usepackage{subcaption}
\usepackage{hyperref}
\usepackage{wrapfig}
\usepackage{fancyhdr}
\usepackage{amsthm}
\usepackage{amsmath}
\usepackage{amssymb}
\usepackage{amsfonts}
\usepackage{cancel}
\usepackage{tcolorbox}


% forza le footnote a stare il più in basso possibile
\usepackage[bottom]{footmisc}

% stile teoremi
\newtheorem{theorem}{Teorema}[section]
\newtheorem{corollary}{Corollario}[theorem]
\newtheorem{lemma}[theorem]{Lemma}
\newtheorem{axch}{Assioma}[section]
\newtheorem{prop}{Proprietà}[section]
\newtheorem{definition}{Definizione}[section]
\newtheorem{limit}{Limitazione}[section]

\renewcommand*{\proofname}{Dimostrazione}

%% STILE LISTINGS

\usepackage{xcolor}

\definecolor{codegreen}{rgb}{0,0.6,0}
\definecolor{codegray}{rgb}{0.5,0.5,0.5}
\definecolor{codepurple}{rgb}{0.58,0,0.82}
\definecolor{backcolour}{rgb}{0.95,0.95,0.92}

 \lstdefinestyle{mystyle}{
     backgroundcolor=\color{backcolour},   
     commentstyle=\color{codegreen},
     keywordstyle=\bfseries\color{black},
     numberstyle=\tiny\color{codegray},
     stringstyle=\color{codepurple},
     basicstyle=\ttfamily\footnotesize,
     breakatwhitespace=false,         
     breaklines=true,                 
     captionpos=b,                    
     keepspaces=true,                 
     numbers=left,                    
     numbersep=5pt,                  
     showspaces=false,                
     showstringspaces=false,
     showtabs=false,                  
     tabsize=2,
     mathescape=true,
     escapeinside={\%*}{*)},
     morekeywords={
        begin,
        end,
        if,
        then,
        else,
        to,
        endif,
        Procedure,
        while,
        do,
        return,
        true,
        false,
        set,
        for,
        repeat,
        until
        foreach},
 }

 \lstset{style=mystyle}

\renewcommand{\lstlistlistingname}{Elenco dei codici}
\renewcommand{\lstlistingname}{Codice}

\newtcolorbox{definition-box}[1]{%
    colback=codepurple!5!white,%
    colframe=codepurple!75!black,%
    fonttitle=\bfseries,%
    title={#1}
}

% mostra le subsubsection nell'indice
\setcounter{tocdepth}{3}
\setcounter{secnumdepth}{3}

% Resetta la numerazione dei chapter quando
% una nuova part viene creata
\makeatletter
\@addtoreset{chapter}{part}
\makeatother

% Rimuove l'indentazione quando si crea un nuovo paragrafo
\setlength{\parindent}{0pt}

% footer
\pagestyle{fancyplain}
% rimuove la riga nell'header
\fancyhf{} % sets both header and footer to nothing
\renewcommand{\headrulewidth}{0pt}
\fancyfoot[L]{\href{https://github.com/Typing-Monkeys/AppuntiUniversita}{Typing Monkeys}}
\fancyfoot[C]{\emoji{gorilla}}
\fancyfoot[R]{\thepage}

% configurazione emoji
\usepackage{fontspec}
\usepackage{emoji}
\setemojifont{NotoColorEmoji.ttf}[Path=/usr/share/fonts/truetype/noto/]

\setlength {\marginparwidth }{2cm}
\usepackage{todonotes}
\newcommand{\TODO}[2][]
{\todo[size=\scriptsize, color=red, #1]{#2}}

\newcommand{\properties}[1]{\textit{\textbf{Proprietà: }} #1 \vspace{5pt}}
\newcommand{\example}[1]{\textbf{\underline{Esempio:}}  #1}


\begin{document}

\documentclass[a4paper,12 pt]{report}
\usepackage[T1]{fontenc}
\usepackage[utf8]{inputenc}
\usepackage[italian]{babel}
\usepackage{lmodern}
\usepackage{listings}
\usepackage{latexsym}
\usepackage{graphicx}
\usepackage{float}
\usepackage{subcaption}
\usepackage{hyperref}
\usepackage{wrapfig}
\usepackage{fancyhdr}
\usepackage{amsthm}
\usepackage{amsmath}
\usepackage{amssymb}
\usepackage{amsfonts}
\usepackage{cancel}
\usepackage{tcolorbox}


% forza le footnote a stare il più in basso possibile
\usepackage[bottom]{footmisc}

% stile teoremi
\newtheorem{theorem}{Teorema}[section]
\newtheorem{corollary}{Corollario}[theorem]
\newtheorem{lemma}[theorem]{Lemma}
\newtheorem{axch}{Assioma}[section]
\newtheorem{prop}{Proprietà}[section]
\newtheorem{definition}{Definizione}[section]
\newtheorem{limit}{Limitazione}[section]

\renewcommand*{\proofname}{Dimostrazione}

%% STILE LISTINGS

\usepackage{xcolor}

\definecolor{codegreen}{rgb}{0,0.6,0}
\definecolor{codegray}{rgb}{0.5,0.5,0.5}
\definecolor{codepurple}{rgb}{0.58,0,0.82}
\definecolor{backcolour}{rgb}{0.95,0.95,0.92}

 \lstdefinestyle{mystyle}{
     backgroundcolor=\color{backcolour},   
     commentstyle=\color{codegreen},
     keywordstyle=\bfseries\color{black},
     numberstyle=\tiny\color{codegray},
     stringstyle=\color{codepurple},
     basicstyle=\ttfamily\footnotesize,
     breakatwhitespace=false,         
     breaklines=true,                 
     captionpos=b,                    
     keepspaces=true,                 
     numbers=left,                    
     numbersep=5pt,                  
     showspaces=false,                
     showstringspaces=false,
     showtabs=false,                  
     tabsize=2,
     mathescape=true,
     escapeinside={\%*}{*)},
     morekeywords={
        begin,
        end,
        if,
        then,
        else,
        to,
        endif,
        Procedure,
        while,
        do,
        return,
        true,
        false,
        set,
        for,
        repeat,
        until
        foreach},
 }

 \lstset{style=mystyle}

\renewcommand{\lstlistlistingname}{Elenco dei codici}
\renewcommand{\lstlistingname}{Codice}

\newtcolorbox{definition-box}[1]{%
    colback=codepurple!5!white,%
    colframe=codepurple!75!black,%
    fonttitle=\bfseries,%
    title={#1}
}

% mostra le subsubsection nell'indice
\setcounter{tocdepth}{3}
\setcounter{secnumdepth}{3}

% Resetta la numerazione dei chapter quando
% una nuova part viene creata
\makeatletter
\@addtoreset{chapter}{part}
\makeatother

% Rimuove l'indentazione quando si crea un nuovo paragrafo
\setlength{\parindent}{0pt}

% footer
\pagestyle{fancyplain}
% rimuove la riga nell'header
\fancyhf{} % sets both header and footer to nothing
\renewcommand{\headrulewidth}{0pt}
\fancyfoot[L]{\href{https://github.com/Typing-Monkeys/AppuntiUniversita}{Typing Monkeys}}
\fancyfoot[C]{\emoji{gorilla}}
\fancyfoot[R]{\thepage}

% configurazione emoji
\usepackage{fontspec}
\usepackage{emoji}
\setemojifont{NotoColorEmoji.ttf}[Path=/usr/share/fonts/truetype/noto/]

\setlength {\marginparwidth }{2cm}
\usepackage{todonotes}
\newcommand{\TODO}[2][]
{\todo[size=\scriptsize, color=red, #1]{#2}}

\newcommand{\properties}[1]{\textit{\textbf{Proprietà: }} #1 \vspace{5pt}}
\newcommand{\example}[1]{\textbf{\underline{Esempio:}}  #1}


\begin{document}

\documentclass[a4paper,12 pt]{report}
\usepackage[T1]{fontenc}
\usepackage[utf8]{inputenc}
\usepackage[italian]{babel}
\usepackage{lmodern}
\usepackage{listings}
\usepackage{latexsym}
\usepackage{graphicx}
\usepackage{float}
\usepackage{subcaption}
\usepackage{hyperref}
\usepackage{wrapfig}
\usepackage{fancyhdr}
\usepackage{amsthm}
\usepackage{amsmath}
\usepackage{amssymb}
\usepackage{amsfonts}
\usepackage{cancel}
\usepackage{tcolorbox}


% forza le footnote a stare il più in basso possibile
\usepackage[bottom]{footmisc}

% stile teoremi
\newtheorem{theorem}{Teorema}[section]
\newtheorem{corollary}{Corollario}[theorem]
\newtheorem{lemma}[theorem]{Lemma}
\newtheorem{axch}{Assioma}[section]
\newtheorem{prop}{Proprietà}[section]
\newtheorem{definition}{Definizione}[section]
\newtheorem{limit}{Limitazione}[section]

\renewcommand*{\proofname}{Dimostrazione}

%% STILE LISTINGS

\usepackage{xcolor}

\definecolor{codegreen}{rgb}{0,0.6,0}
\definecolor{codegray}{rgb}{0.5,0.5,0.5}
\definecolor{codepurple}{rgb}{0.58,0,0.82}
\definecolor{backcolour}{rgb}{0.95,0.95,0.92}

 \lstdefinestyle{mystyle}{
     backgroundcolor=\color{backcolour},   
     commentstyle=\color{codegreen},
     keywordstyle=\bfseries\color{black},
     numberstyle=\tiny\color{codegray},
     stringstyle=\color{codepurple},
     basicstyle=\ttfamily\footnotesize,
     breakatwhitespace=false,         
     breaklines=true,                 
     captionpos=b,                    
     keepspaces=true,                 
     numbers=left,                    
     numbersep=5pt,                  
     showspaces=false,                
     showstringspaces=false,
     showtabs=false,                  
     tabsize=2,
     mathescape=true,
     escapeinside={\%*}{*)},
     morekeywords={
        begin,
        end,
        if,
        then,
        else,
        to,
        endif,
        Procedure,
        while,
        do,
        return,
        true,
        false,
        set,
        for,
        repeat,
        until
        foreach},
 }

 \lstset{style=mystyle}

\renewcommand{\lstlistlistingname}{Elenco dei codici}
\renewcommand{\lstlistingname}{Codice}

\newtcolorbox{definition-box}[1]{%
    colback=codepurple!5!white,%
    colframe=codepurple!75!black,%
    fonttitle=\bfseries,%
    title={#1}
}

% mostra le subsubsection nell'indice
\setcounter{tocdepth}{3}
\setcounter{secnumdepth}{3}

% Resetta la numerazione dei chapter quando
% una nuova part viene creata
\makeatletter
\@addtoreset{chapter}{part}
\makeatother

% Rimuove l'indentazione quando si crea un nuovo paragrafo
\setlength{\parindent}{0pt}

% footer
\pagestyle{fancyplain}
% rimuove la riga nell'header
\fancyhf{} % sets both header and footer to nothing
\renewcommand{\headrulewidth}{0pt}
\fancyfoot[L]{\href{https://github.com/Typing-Monkeys/AppuntiUniversita}{Typing Monkeys}}
\fancyfoot[C]{\emoji{gorilla}}
\fancyfoot[R]{\thepage}

% configurazione emoji
\usepackage{fontspec}
\usepackage{emoji}
\setemojifont{NotoColorEmoji.ttf}[Path=/usr/share/fonts/truetype/noto/]

\setlength {\marginparwidth }{2cm}
\usepackage{todonotes}
\newcommand{\TODO}[2][]
{\todo[size=\scriptsize, color=red, #1]{#2}}

\newcommand{\properties}[1]{\textit{\textbf{Proprietà: }} #1 \vspace{5pt}}
\newcommand{\example}[1]{\textbf{\underline{Esempio:}}  #1}


\begin{document}

\include{frontmatter/main.tex}

%% Aggiungere i capitoli qui sotto
\include{capitoli/introduzione/main}
\include{capitoli/agenti-intelligenti/main}


\end{document}

%% Aggiungere i capitoli qui sotto
\documentclass[a4paper,12 pt]{report}
\usepackage[T1]{fontenc}
\usepackage[utf8]{inputenc}
\usepackage[italian]{babel}
\usepackage{lmodern}
\usepackage{listings}
\usepackage{latexsym}
\usepackage{graphicx}
\usepackage{float}
\usepackage{subcaption}
\usepackage{hyperref}
\usepackage{wrapfig}
\usepackage{fancyhdr}
\usepackage{amsthm}
\usepackage{amsmath}
\usepackage{amssymb}
\usepackage{amsfonts}
\usepackage{cancel}
\usepackage{tcolorbox}


% forza le footnote a stare il più in basso possibile
\usepackage[bottom]{footmisc}

% stile teoremi
\newtheorem{theorem}{Teorema}[section]
\newtheorem{corollary}{Corollario}[theorem]
\newtheorem{lemma}[theorem]{Lemma}
\newtheorem{axch}{Assioma}[section]
\newtheorem{prop}{Proprietà}[section]
\newtheorem{definition}{Definizione}[section]
\newtheorem{limit}{Limitazione}[section]

\renewcommand*{\proofname}{Dimostrazione}

%% STILE LISTINGS

\usepackage{xcolor}

\definecolor{codegreen}{rgb}{0,0.6,0}
\definecolor{codegray}{rgb}{0.5,0.5,0.5}
\definecolor{codepurple}{rgb}{0.58,0,0.82}
\definecolor{backcolour}{rgb}{0.95,0.95,0.92}

 \lstdefinestyle{mystyle}{
     backgroundcolor=\color{backcolour},   
     commentstyle=\color{codegreen},
     keywordstyle=\bfseries\color{black},
     numberstyle=\tiny\color{codegray},
     stringstyle=\color{codepurple},
     basicstyle=\ttfamily\footnotesize,
     breakatwhitespace=false,         
     breaklines=true,                 
     captionpos=b,                    
     keepspaces=true,                 
     numbers=left,                    
     numbersep=5pt,                  
     showspaces=false,                
     showstringspaces=false,
     showtabs=false,                  
     tabsize=2,
     mathescape=true,
     escapeinside={\%*}{*)},
     morekeywords={
        begin,
        end,
        if,
        then,
        else,
        to,
        endif,
        Procedure,
        while,
        do,
        return,
        true,
        false,
        set,
        for,
        repeat,
        until
        foreach},
 }

 \lstset{style=mystyle}

\renewcommand{\lstlistlistingname}{Elenco dei codici}
\renewcommand{\lstlistingname}{Codice}

\newtcolorbox{definition-box}[1]{%
    colback=codepurple!5!white,%
    colframe=codepurple!75!black,%
    fonttitle=\bfseries,%
    title={#1}
}

% mostra le subsubsection nell'indice
\setcounter{tocdepth}{3}
\setcounter{secnumdepth}{3}

% Resetta la numerazione dei chapter quando
% una nuova part viene creata
\makeatletter
\@addtoreset{chapter}{part}
\makeatother

% Rimuove l'indentazione quando si crea un nuovo paragrafo
\setlength{\parindent}{0pt}

% footer
\pagestyle{fancyplain}
% rimuove la riga nell'header
\fancyhf{} % sets both header and footer to nothing
\renewcommand{\headrulewidth}{0pt}
\fancyfoot[L]{\href{https://github.com/Typing-Monkeys/AppuntiUniversita}{Typing Monkeys}}
\fancyfoot[C]{\emoji{gorilla}}
\fancyfoot[R]{\thepage}

% configurazione emoji
\usepackage{fontspec}
\usepackage{emoji}
\setemojifont{NotoColorEmoji.ttf}[Path=/usr/share/fonts/truetype/noto/]

\setlength {\marginparwidth }{2cm}
\usepackage{todonotes}
\newcommand{\TODO}[2][]
{\todo[size=\scriptsize, color=red, #1]{#2}}

\newcommand{\properties}[1]{\textit{\textbf{Proprietà: }} #1 \vspace{5pt}}
\newcommand{\example}[1]{\textbf{\underline{Esempio:}}  #1}


\begin{document}

\include{frontmatter/main.tex}

%% Aggiungere i capitoli qui sotto
\include{capitoli/introduzione/main}
\include{capitoli/agenti-intelligenti/main}


\end{document}
\documentclass[a4paper,12 pt]{report}
\usepackage[T1]{fontenc}
\usepackage[utf8]{inputenc}
\usepackage[italian]{babel}
\usepackage{lmodern}
\usepackage{listings}
\usepackage{latexsym}
\usepackage{graphicx}
\usepackage{float}
\usepackage{subcaption}
\usepackage{hyperref}
\usepackage{wrapfig}
\usepackage{fancyhdr}
\usepackage{amsthm}
\usepackage{amsmath}
\usepackage{amssymb}
\usepackage{amsfonts}
\usepackage{cancel}
\usepackage{tcolorbox}


% forza le footnote a stare il più in basso possibile
\usepackage[bottom]{footmisc}

% stile teoremi
\newtheorem{theorem}{Teorema}[section]
\newtheorem{corollary}{Corollario}[theorem]
\newtheorem{lemma}[theorem]{Lemma}
\newtheorem{axch}{Assioma}[section]
\newtheorem{prop}{Proprietà}[section]
\newtheorem{definition}{Definizione}[section]
\newtheorem{limit}{Limitazione}[section]

\renewcommand*{\proofname}{Dimostrazione}

%% STILE LISTINGS

\usepackage{xcolor}

\definecolor{codegreen}{rgb}{0,0.6,0}
\definecolor{codegray}{rgb}{0.5,0.5,0.5}
\definecolor{codepurple}{rgb}{0.58,0,0.82}
\definecolor{backcolour}{rgb}{0.95,0.95,0.92}

 \lstdefinestyle{mystyle}{
     backgroundcolor=\color{backcolour},   
     commentstyle=\color{codegreen},
     keywordstyle=\bfseries\color{black},
     numberstyle=\tiny\color{codegray},
     stringstyle=\color{codepurple},
     basicstyle=\ttfamily\footnotesize,
     breakatwhitespace=false,         
     breaklines=true,                 
     captionpos=b,                    
     keepspaces=true,                 
     numbers=left,                    
     numbersep=5pt,                  
     showspaces=false,                
     showstringspaces=false,
     showtabs=false,                  
     tabsize=2,
     mathescape=true,
     escapeinside={\%*}{*)},
     morekeywords={
        begin,
        end,
        if,
        then,
        else,
        to,
        endif,
        Procedure,
        while,
        do,
        return,
        true,
        false,
        set,
        for,
        repeat,
        until
        foreach},
 }

 \lstset{style=mystyle}

\renewcommand{\lstlistlistingname}{Elenco dei codici}
\renewcommand{\lstlistingname}{Codice}

\newtcolorbox{definition-box}[1]{%
    colback=codepurple!5!white,%
    colframe=codepurple!75!black,%
    fonttitle=\bfseries,%
    title={#1}
}

% mostra le subsubsection nell'indice
\setcounter{tocdepth}{3}
\setcounter{secnumdepth}{3}

% Resetta la numerazione dei chapter quando
% una nuova part viene creata
\makeatletter
\@addtoreset{chapter}{part}
\makeatother

% Rimuove l'indentazione quando si crea un nuovo paragrafo
\setlength{\parindent}{0pt}

% footer
\pagestyle{fancyplain}
% rimuove la riga nell'header
\fancyhf{} % sets both header and footer to nothing
\renewcommand{\headrulewidth}{0pt}
\fancyfoot[L]{\href{https://github.com/Typing-Monkeys/AppuntiUniversita}{Typing Monkeys}}
\fancyfoot[C]{\emoji{gorilla}}
\fancyfoot[R]{\thepage}

% configurazione emoji
\usepackage{fontspec}
\usepackage{emoji}
\setemojifont{NotoColorEmoji.ttf}[Path=/usr/share/fonts/truetype/noto/]

\setlength {\marginparwidth }{2cm}
\usepackage{todonotes}
\newcommand{\TODO}[2][]
{\todo[size=\scriptsize, color=red, #1]{#2}}

\newcommand{\properties}[1]{\textit{\textbf{Proprietà: }} #1 \vspace{5pt}}
\newcommand{\example}[1]{\textbf{\underline{Esempio:}}  #1}


\begin{document}

\include{frontmatter/main.tex}

%% Aggiungere i capitoli qui sotto
\include{capitoli/introduzione/main}
\include{capitoli/agenti-intelligenti/main}


\end{document}


\end{document}

%% Aggiungere i capitoli qui sotto
\documentclass[a4paper,12 pt]{report}
\usepackage[T1]{fontenc}
\usepackage[utf8]{inputenc}
\usepackage[italian]{babel}
\usepackage{lmodern}
\usepackage{listings}
\usepackage{latexsym}
\usepackage{graphicx}
\usepackage{float}
\usepackage{subcaption}
\usepackage{hyperref}
\usepackage{wrapfig}
\usepackage{fancyhdr}
\usepackage{amsthm}
\usepackage{amsmath}
\usepackage{amssymb}
\usepackage{amsfonts}
\usepackage{cancel}
\usepackage{tcolorbox}


% forza le footnote a stare il più in basso possibile
\usepackage[bottom]{footmisc}

% stile teoremi
\newtheorem{theorem}{Teorema}[section]
\newtheorem{corollary}{Corollario}[theorem]
\newtheorem{lemma}[theorem]{Lemma}
\newtheorem{axch}{Assioma}[section]
\newtheorem{prop}{Proprietà}[section]
\newtheorem{definition}{Definizione}[section]
\newtheorem{limit}{Limitazione}[section]

\renewcommand*{\proofname}{Dimostrazione}

%% STILE LISTINGS

\usepackage{xcolor}

\definecolor{codegreen}{rgb}{0,0.6,0}
\definecolor{codegray}{rgb}{0.5,0.5,0.5}
\definecolor{codepurple}{rgb}{0.58,0,0.82}
\definecolor{backcolour}{rgb}{0.95,0.95,0.92}

 \lstdefinestyle{mystyle}{
     backgroundcolor=\color{backcolour},   
     commentstyle=\color{codegreen},
     keywordstyle=\bfseries\color{black},
     numberstyle=\tiny\color{codegray},
     stringstyle=\color{codepurple},
     basicstyle=\ttfamily\footnotesize,
     breakatwhitespace=false,         
     breaklines=true,                 
     captionpos=b,                    
     keepspaces=true,                 
     numbers=left,                    
     numbersep=5pt,                  
     showspaces=false,                
     showstringspaces=false,
     showtabs=false,                  
     tabsize=2,
     mathescape=true,
     escapeinside={\%*}{*)},
     morekeywords={
        begin,
        end,
        if,
        then,
        else,
        to,
        endif,
        Procedure,
        while,
        do,
        return,
        true,
        false,
        set,
        for,
        repeat,
        until
        foreach},
 }

 \lstset{style=mystyle}

\renewcommand{\lstlistlistingname}{Elenco dei codici}
\renewcommand{\lstlistingname}{Codice}

\newtcolorbox{definition-box}[1]{%
    colback=codepurple!5!white,%
    colframe=codepurple!75!black,%
    fonttitle=\bfseries,%
    title={#1}
}

% mostra le subsubsection nell'indice
\setcounter{tocdepth}{3}
\setcounter{secnumdepth}{3}

% Resetta la numerazione dei chapter quando
% una nuova part viene creata
\makeatletter
\@addtoreset{chapter}{part}
\makeatother

% Rimuove l'indentazione quando si crea un nuovo paragrafo
\setlength{\parindent}{0pt}

% footer
\pagestyle{fancyplain}
% rimuove la riga nell'header
\fancyhf{} % sets both header and footer to nothing
\renewcommand{\headrulewidth}{0pt}
\fancyfoot[L]{\href{https://github.com/Typing-Monkeys/AppuntiUniversita}{Typing Monkeys}}
\fancyfoot[C]{\emoji{gorilla}}
\fancyfoot[R]{\thepage}

% configurazione emoji
\usepackage{fontspec}
\usepackage{emoji}
\setemojifont{NotoColorEmoji.ttf}[Path=/usr/share/fonts/truetype/noto/]

\setlength {\marginparwidth }{2cm}
\usepackage{todonotes}
\newcommand{\TODO}[2][]
{\todo[size=\scriptsize, color=red, #1]{#2}}

\newcommand{\properties}[1]{\textit{\textbf{Proprietà: }} #1 \vspace{5pt}}
\newcommand{\example}[1]{\textbf{\underline{Esempio:}}  #1}


\begin{document}

\documentclass[a4paper,12 pt]{report}
\usepackage[T1]{fontenc}
\usepackage[utf8]{inputenc}
\usepackage[italian]{babel}
\usepackage{lmodern}
\usepackage{listings}
\usepackage{latexsym}
\usepackage{graphicx}
\usepackage{float}
\usepackage{subcaption}
\usepackage{hyperref}
\usepackage{wrapfig}
\usepackage{fancyhdr}
\usepackage{amsthm}
\usepackage{amsmath}
\usepackage{amssymb}
\usepackage{amsfonts}
\usepackage{cancel}
\usepackage{tcolorbox}


% forza le footnote a stare il più in basso possibile
\usepackage[bottom]{footmisc}

% stile teoremi
\newtheorem{theorem}{Teorema}[section]
\newtheorem{corollary}{Corollario}[theorem]
\newtheorem{lemma}[theorem]{Lemma}
\newtheorem{axch}{Assioma}[section]
\newtheorem{prop}{Proprietà}[section]
\newtheorem{definition}{Definizione}[section]
\newtheorem{limit}{Limitazione}[section]

\renewcommand*{\proofname}{Dimostrazione}

%% STILE LISTINGS

\usepackage{xcolor}

\definecolor{codegreen}{rgb}{0,0.6,0}
\definecolor{codegray}{rgb}{0.5,0.5,0.5}
\definecolor{codepurple}{rgb}{0.58,0,0.82}
\definecolor{backcolour}{rgb}{0.95,0.95,0.92}

 \lstdefinestyle{mystyle}{
     backgroundcolor=\color{backcolour},   
     commentstyle=\color{codegreen},
     keywordstyle=\bfseries\color{black},
     numberstyle=\tiny\color{codegray},
     stringstyle=\color{codepurple},
     basicstyle=\ttfamily\footnotesize,
     breakatwhitespace=false,         
     breaklines=true,                 
     captionpos=b,                    
     keepspaces=true,                 
     numbers=left,                    
     numbersep=5pt,                  
     showspaces=false,                
     showstringspaces=false,
     showtabs=false,                  
     tabsize=2,
     mathescape=true,
     escapeinside={\%*}{*)},
     morekeywords={
        begin,
        end,
        if,
        then,
        else,
        to,
        endif,
        Procedure,
        while,
        do,
        return,
        true,
        false,
        set,
        for,
        repeat,
        until
        foreach},
 }

 \lstset{style=mystyle}

\renewcommand{\lstlistlistingname}{Elenco dei codici}
\renewcommand{\lstlistingname}{Codice}

\newtcolorbox{definition-box}[1]{%
    colback=codepurple!5!white,%
    colframe=codepurple!75!black,%
    fonttitle=\bfseries,%
    title={#1}
}

% mostra le subsubsection nell'indice
\setcounter{tocdepth}{3}
\setcounter{secnumdepth}{3}

% Resetta la numerazione dei chapter quando
% una nuova part viene creata
\makeatletter
\@addtoreset{chapter}{part}
\makeatother

% Rimuove l'indentazione quando si crea un nuovo paragrafo
\setlength{\parindent}{0pt}

% footer
\pagestyle{fancyplain}
% rimuove la riga nell'header
\fancyhf{} % sets both header and footer to nothing
\renewcommand{\headrulewidth}{0pt}
\fancyfoot[L]{\href{https://github.com/Typing-Monkeys/AppuntiUniversita}{Typing Monkeys}}
\fancyfoot[C]{\emoji{gorilla}}
\fancyfoot[R]{\thepage}

% configurazione emoji
\usepackage{fontspec}
\usepackage{emoji}
\setemojifont{NotoColorEmoji.ttf}[Path=/usr/share/fonts/truetype/noto/]

\setlength {\marginparwidth }{2cm}
\usepackage{todonotes}
\newcommand{\TODO}[2][]
{\todo[size=\scriptsize, color=red, #1]{#2}}

\newcommand{\properties}[1]{\textit{\textbf{Proprietà: }} #1 \vspace{5pt}}
\newcommand{\example}[1]{\textbf{\underline{Esempio:}}  #1}


\begin{document}

\include{frontmatter/main.tex}

%% Aggiungere i capitoli qui sotto
\include{capitoli/introduzione/main}
\include{capitoli/agenti-intelligenti/main}


\end{document}

%% Aggiungere i capitoli qui sotto
\documentclass[a4paper,12 pt]{report}
\usepackage[T1]{fontenc}
\usepackage[utf8]{inputenc}
\usepackage[italian]{babel}
\usepackage{lmodern}
\usepackage{listings}
\usepackage{latexsym}
\usepackage{graphicx}
\usepackage{float}
\usepackage{subcaption}
\usepackage{hyperref}
\usepackage{wrapfig}
\usepackage{fancyhdr}
\usepackage{amsthm}
\usepackage{amsmath}
\usepackage{amssymb}
\usepackage{amsfonts}
\usepackage{cancel}
\usepackage{tcolorbox}


% forza le footnote a stare il più in basso possibile
\usepackage[bottom]{footmisc}

% stile teoremi
\newtheorem{theorem}{Teorema}[section]
\newtheorem{corollary}{Corollario}[theorem]
\newtheorem{lemma}[theorem]{Lemma}
\newtheorem{axch}{Assioma}[section]
\newtheorem{prop}{Proprietà}[section]
\newtheorem{definition}{Definizione}[section]
\newtheorem{limit}{Limitazione}[section]

\renewcommand*{\proofname}{Dimostrazione}

%% STILE LISTINGS

\usepackage{xcolor}

\definecolor{codegreen}{rgb}{0,0.6,0}
\definecolor{codegray}{rgb}{0.5,0.5,0.5}
\definecolor{codepurple}{rgb}{0.58,0,0.82}
\definecolor{backcolour}{rgb}{0.95,0.95,0.92}

 \lstdefinestyle{mystyle}{
     backgroundcolor=\color{backcolour},   
     commentstyle=\color{codegreen},
     keywordstyle=\bfseries\color{black},
     numberstyle=\tiny\color{codegray},
     stringstyle=\color{codepurple},
     basicstyle=\ttfamily\footnotesize,
     breakatwhitespace=false,         
     breaklines=true,                 
     captionpos=b,                    
     keepspaces=true,                 
     numbers=left,                    
     numbersep=5pt,                  
     showspaces=false,                
     showstringspaces=false,
     showtabs=false,                  
     tabsize=2,
     mathescape=true,
     escapeinside={\%*}{*)},
     morekeywords={
        begin,
        end,
        if,
        then,
        else,
        to,
        endif,
        Procedure,
        while,
        do,
        return,
        true,
        false,
        set,
        for,
        repeat,
        until
        foreach},
 }

 \lstset{style=mystyle}

\renewcommand{\lstlistlistingname}{Elenco dei codici}
\renewcommand{\lstlistingname}{Codice}

\newtcolorbox{definition-box}[1]{%
    colback=codepurple!5!white,%
    colframe=codepurple!75!black,%
    fonttitle=\bfseries,%
    title={#1}
}

% mostra le subsubsection nell'indice
\setcounter{tocdepth}{3}
\setcounter{secnumdepth}{3}

% Resetta la numerazione dei chapter quando
% una nuova part viene creata
\makeatletter
\@addtoreset{chapter}{part}
\makeatother

% Rimuove l'indentazione quando si crea un nuovo paragrafo
\setlength{\parindent}{0pt}

% footer
\pagestyle{fancyplain}
% rimuove la riga nell'header
\fancyhf{} % sets both header and footer to nothing
\renewcommand{\headrulewidth}{0pt}
\fancyfoot[L]{\href{https://github.com/Typing-Monkeys/AppuntiUniversita}{Typing Monkeys}}
\fancyfoot[C]{\emoji{gorilla}}
\fancyfoot[R]{\thepage}

% configurazione emoji
\usepackage{fontspec}
\usepackage{emoji}
\setemojifont{NotoColorEmoji.ttf}[Path=/usr/share/fonts/truetype/noto/]

\setlength {\marginparwidth }{2cm}
\usepackage{todonotes}
\newcommand{\TODO}[2][]
{\todo[size=\scriptsize, color=red, #1]{#2}}

\newcommand{\properties}[1]{\textit{\textbf{Proprietà: }} #1 \vspace{5pt}}
\newcommand{\example}[1]{\textbf{\underline{Esempio:}}  #1}


\begin{document}

\include{frontmatter/main.tex}

%% Aggiungere i capitoli qui sotto
\include{capitoli/introduzione/main}
\include{capitoli/agenti-intelligenti/main}


\end{document}
\documentclass[a4paper,12 pt]{report}
\usepackage[T1]{fontenc}
\usepackage[utf8]{inputenc}
\usepackage[italian]{babel}
\usepackage{lmodern}
\usepackage{listings}
\usepackage{latexsym}
\usepackage{graphicx}
\usepackage{float}
\usepackage{subcaption}
\usepackage{hyperref}
\usepackage{wrapfig}
\usepackage{fancyhdr}
\usepackage{amsthm}
\usepackage{amsmath}
\usepackage{amssymb}
\usepackage{amsfonts}
\usepackage{cancel}
\usepackage{tcolorbox}


% forza le footnote a stare il più in basso possibile
\usepackage[bottom]{footmisc}

% stile teoremi
\newtheorem{theorem}{Teorema}[section]
\newtheorem{corollary}{Corollario}[theorem]
\newtheorem{lemma}[theorem]{Lemma}
\newtheorem{axch}{Assioma}[section]
\newtheorem{prop}{Proprietà}[section]
\newtheorem{definition}{Definizione}[section]
\newtheorem{limit}{Limitazione}[section]

\renewcommand*{\proofname}{Dimostrazione}

%% STILE LISTINGS

\usepackage{xcolor}

\definecolor{codegreen}{rgb}{0,0.6,0}
\definecolor{codegray}{rgb}{0.5,0.5,0.5}
\definecolor{codepurple}{rgb}{0.58,0,0.82}
\definecolor{backcolour}{rgb}{0.95,0.95,0.92}

 \lstdefinestyle{mystyle}{
     backgroundcolor=\color{backcolour},   
     commentstyle=\color{codegreen},
     keywordstyle=\bfseries\color{black},
     numberstyle=\tiny\color{codegray},
     stringstyle=\color{codepurple},
     basicstyle=\ttfamily\footnotesize,
     breakatwhitespace=false,         
     breaklines=true,                 
     captionpos=b,                    
     keepspaces=true,                 
     numbers=left,                    
     numbersep=5pt,                  
     showspaces=false,                
     showstringspaces=false,
     showtabs=false,                  
     tabsize=2,
     mathescape=true,
     escapeinside={\%*}{*)},
     morekeywords={
        begin,
        end,
        if,
        then,
        else,
        to,
        endif,
        Procedure,
        while,
        do,
        return,
        true,
        false,
        set,
        for,
        repeat,
        until
        foreach},
 }

 \lstset{style=mystyle}

\renewcommand{\lstlistlistingname}{Elenco dei codici}
\renewcommand{\lstlistingname}{Codice}

\newtcolorbox{definition-box}[1]{%
    colback=codepurple!5!white,%
    colframe=codepurple!75!black,%
    fonttitle=\bfseries,%
    title={#1}
}

% mostra le subsubsection nell'indice
\setcounter{tocdepth}{3}
\setcounter{secnumdepth}{3}

% Resetta la numerazione dei chapter quando
% una nuova part viene creata
\makeatletter
\@addtoreset{chapter}{part}
\makeatother

% Rimuove l'indentazione quando si crea un nuovo paragrafo
\setlength{\parindent}{0pt}

% footer
\pagestyle{fancyplain}
% rimuove la riga nell'header
\fancyhf{} % sets both header and footer to nothing
\renewcommand{\headrulewidth}{0pt}
\fancyfoot[L]{\href{https://github.com/Typing-Monkeys/AppuntiUniversita}{Typing Monkeys}}
\fancyfoot[C]{\emoji{gorilla}}
\fancyfoot[R]{\thepage}

% configurazione emoji
\usepackage{fontspec}
\usepackage{emoji}
\setemojifont{NotoColorEmoji.ttf}[Path=/usr/share/fonts/truetype/noto/]

\setlength {\marginparwidth }{2cm}
\usepackage{todonotes}
\newcommand{\TODO}[2][]
{\todo[size=\scriptsize, color=red, #1]{#2}}

\newcommand{\properties}[1]{\textit{\textbf{Proprietà: }} #1 \vspace{5pt}}
\newcommand{\example}[1]{\textbf{\underline{Esempio:}}  #1}


\begin{document}

\include{frontmatter/main.tex}

%% Aggiungere i capitoli qui sotto
\include{capitoli/introduzione/main}
\include{capitoli/agenti-intelligenti/main}


\end{document}


\end{document}
\documentclass[a4paper,12 pt]{report}
\usepackage[T1]{fontenc}
\usepackage[utf8]{inputenc}
\usepackage[italian]{babel}
\usepackage{lmodern}
\usepackage{listings}
\usepackage{latexsym}
\usepackage{graphicx}
\usepackage{float}
\usepackage{subcaption}
\usepackage{hyperref}
\usepackage{wrapfig}
\usepackage{fancyhdr}
\usepackage{amsthm}
\usepackage{amsmath}
\usepackage{amssymb}
\usepackage{amsfonts}
\usepackage{cancel}
\usepackage{tcolorbox}


% forza le footnote a stare il più in basso possibile
\usepackage[bottom]{footmisc}

% stile teoremi
\newtheorem{theorem}{Teorema}[section]
\newtheorem{corollary}{Corollario}[theorem]
\newtheorem{lemma}[theorem]{Lemma}
\newtheorem{axch}{Assioma}[section]
\newtheorem{prop}{Proprietà}[section]
\newtheorem{definition}{Definizione}[section]
\newtheorem{limit}{Limitazione}[section]

\renewcommand*{\proofname}{Dimostrazione}

%% STILE LISTINGS

\usepackage{xcolor}

\definecolor{codegreen}{rgb}{0,0.6,0}
\definecolor{codegray}{rgb}{0.5,0.5,0.5}
\definecolor{codepurple}{rgb}{0.58,0,0.82}
\definecolor{backcolour}{rgb}{0.95,0.95,0.92}

 \lstdefinestyle{mystyle}{
     backgroundcolor=\color{backcolour},   
     commentstyle=\color{codegreen},
     keywordstyle=\bfseries\color{black},
     numberstyle=\tiny\color{codegray},
     stringstyle=\color{codepurple},
     basicstyle=\ttfamily\footnotesize,
     breakatwhitespace=false,         
     breaklines=true,                 
     captionpos=b,                    
     keepspaces=true,                 
     numbers=left,                    
     numbersep=5pt,                  
     showspaces=false,                
     showstringspaces=false,
     showtabs=false,                  
     tabsize=2,
     mathescape=true,
     escapeinside={\%*}{*)},
     morekeywords={
        begin,
        end,
        if,
        then,
        else,
        to,
        endif,
        Procedure,
        while,
        do,
        return,
        true,
        false,
        set,
        for,
        repeat,
        until
        foreach},
 }

 \lstset{style=mystyle}

\renewcommand{\lstlistlistingname}{Elenco dei codici}
\renewcommand{\lstlistingname}{Codice}

\newtcolorbox{definition-box}[1]{%
    colback=codepurple!5!white,%
    colframe=codepurple!75!black,%
    fonttitle=\bfseries,%
    title={#1}
}

% mostra le subsubsection nell'indice
\setcounter{tocdepth}{3}
\setcounter{secnumdepth}{3}

% Resetta la numerazione dei chapter quando
% una nuova part viene creata
\makeatletter
\@addtoreset{chapter}{part}
\makeatother

% Rimuove l'indentazione quando si crea un nuovo paragrafo
\setlength{\parindent}{0pt}

% footer
\pagestyle{fancyplain}
% rimuove la riga nell'header
\fancyhf{} % sets both header and footer to nothing
\renewcommand{\headrulewidth}{0pt}
\fancyfoot[L]{\href{https://github.com/Typing-Monkeys/AppuntiUniversita}{Typing Monkeys}}
\fancyfoot[C]{\emoji{gorilla}}
\fancyfoot[R]{\thepage}

% configurazione emoji
\usepackage{fontspec}
\usepackage{emoji}
\setemojifont{NotoColorEmoji.ttf}[Path=/usr/share/fonts/truetype/noto/]

\setlength {\marginparwidth }{2cm}
\usepackage{todonotes}
\newcommand{\TODO}[2][]
{\todo[size=\scriptsize, color=red, #1]{#2}}

\newcommand{\properties}[1]{\textit{\textbf{Proprietà: }} #1 \vspace{5pt}}
\newcommand{\example}[1]{\textbf{\underline{Esempio:}}  #1}


\begin{document}

\documentclass[a4paper,12 pt]{report}
\usepackage[T1]{fontenc}
\usepackage[utf8]{inputenc}
\usepackage[italian]{babel}
\usepackage{lmodern}
\usepackage{listings}
\usepackage{latexsym}
\usepackage{graphicx}
\usepackage{float}
\usepackage{subcaption}
\usepackage{hyperref}
\usepackage{wrapfig}
\usepackage{fancyhdr}
\usepackage{amsthm}
\usepackage{amsmath}
\usepackage{amssymb}
\usepackage{amsfonts}
\usepackage{cancel}
\usepackage{tcolorbox}


% forza le footnote a stare il più in basso possibile
\usepackage[bottom]{footmisc}

% stile teoremi
\newtheorem{theorem}{Teorema}[section]
\newtheorem{corollary}{Corollario}[theorem]
\newtheorem{lemma}[theorem]{Lemma}
\newtheorem{axch}{Assioma}[section]
\newtheorem{prop}{Proprietà}[section]
\newtheorem{definition}{Definizione}[section]
\newtheorem{limit}{Limitazione}[section]

\renewcommand*{\proofname}{Dimostrazione}

%% STILE LISTINGS

\usepackage{xcolor}

\definecolor{codegreen}{rgb}{0,0.6,0}
\definecolor{codegray}{rgb}{0.5,0.5,0.5}
\definecolor{codepurple}{rgb}{0.58,0,0.82}
\definecolor{backcolour}{rgb}{0.95,0.95,0.92}

 \lstdefinestyle{mystyle}{
     backgroundcolor=\color{backcolour},   
     commentstyle=\color{codegreen},
     keywordstyle=\bfseries\color{black},
     numberstyle=\tiny\color{codegray},
     stringstyle=\color{codepurple},
     basicstyle=\ttfamily\footnotesize,
     breakatwhitespace=false,         
     breaklines=true,                 
     captionpos=b,                    
     keepspaces=true,                 
     numbers=left,                    
     numbersep=5pt,                  
     showspaces=false,                
     showstringspaces=false,
     showtabs=false,                  
     tabsize=2,
     mathescape=true,
     escapeinside={\%*}{*)},
     morekeywords={
        begin,
        end,
        if,
        then,
        else,
        to,
        endif,
        Procedure,
        while,
        do,
        return,
        true,
        false,
        set,
        for,
        repeat,
        until
        foreach},
 }

 \lstset{style=mystyle}

\renewcommand{\lstlistlistingname}{Elenco dei codici}
\renewcommand{\lstlistingname}{Codice}

\newtcolorbox{definition-box}[1]{%
    colback=codepurple!5!white,%
    colframe=codepurple!75!black,%
    fonttitle=\bfseries,%
    title={#1}
}

% mostra le subsubsection nell'indice
\setcounter{tocdepth}{3}
\setcounter{secnumdepth}{3}

% Resetta la numerazione dei chapter quando
% una nuova part viene creata
\makeatletter
\@addtoreset{chapter}{part}
\makeatother

% Rimuove l'indentazione quando si crea un nuovo paragrafo
\setlength{\parindent}{0pt}

% footer
\pagestyle{fancyplain}
% rimuove la riga nell'header
\fancyhf{} % sets both header and footer to nothing
\renewcommand{\headrulewidth}{0pt}
\fancyfoot[L]{\href{https://github.com/Typing-Monkeys/AppuntiUniversita}{Typing Monkeys}}
\fancyfoot[C]{\emoji{gorilla}}
\fancyfoot[R]{\thepage}

% configurazione emoji
\usepackage{fontspec}
\usepackage{emoji}
\setemojifont{NotoColorEmoji.ttf}[Path=/usr/share/fonts/truetype/noto/]

\setlength {\marginparwidth }{2cm}
\usepackage{todonotes}
\newcommand{\TODO}[2][]
{\todo[size=\scriptsize, color=red, #1]{#2}}

\newcommand{\properties}[1]{\textit{\textbf{Proprietà: }} #1 \vspace{5pt}}
\newcommand{\example}[1]{\textbf{\underline{Esempio:}}  #1}


\begin{document}

\include{frontmatter/main.tex}

%% Aggiungere i capitoli qui sotto
\include{capitoli/introduzione/main}
\include{capitoli/agenti-intelligenti/main}


\end{document}

%% Aggiungere i capitoli qui sotto
\documentclass[a4paper,12 pt]{report}
\usepackage[T1]{fontenc}
\usepackage[utf8]{inputenc}
\usepackage[italian]{babel}
\usepackage{lmodern}
\usepackage{listings}
\usepackage{latexsym}
\usepackage{graphicx}
\usepackage{float}
\usepackage{subcaption}
\usepackage{hyperref}
\usepackage{wrapfig}
\usepackage{fancyhdr}
\usepackage{amsthm}
\usepackage{amsmath}
\usepackage{amssymb}
\usepackage{amsfonts}
\usepackage{cancel}
\usepackage{tcolorbox}


% forza le footnote a stare il più in basso possibile
\usepackage[bottom]{footmisc}

% stile teoremi
\newtheorem{theorem}{Teorema}[section]
\newtheorem{corollary}{Corollario}[theorem]
\newtheorem{lemma}[theorem]{Lemma}
\newtheorem{axch}{Assioma}[section]
\newtheorem{prop}{Proprietà}[section]
\newtheorem{definition}{Definizione}[section]
\newtheorem{limit}{Limitazione}[section]

\renewcommand*{\proofname}{Dimostrazione}

%% STILE LISTINGS

\usepackage{xcolor}

\definecolor{codegreen}{rgb}{0,0.6,0}
\definecolor{codegray}{rgb}{0.5,0.5,0.5}
\definecolor{codepurple}{rgb}{0.58,0,0.82}
\definecolor{backcolour}{rgb}{0.95,0.95,0.92}

 \lstdefinestyle{mystyle}{
     backgroundcolor=\color{backcolour},   
     commentstyle=\color{codegreen},
     keywordstyle=\bfseries\color{black},
     numberstyle=\tiny\color{codegray},
     stringstyle=\color{codepurple},
     basicstyle=\ttfamily\footnotesize,
     breakatwhitespace=false,         
     breaklines=true,                 
     captionpos=b,                    
     keepspaces=true,                 
     numbers=left,                    
     numbersep=5pt,                  
     showspaces=false,                
     showstringspaces=false,
     showtabs=false,                  
     tabsize=2,
     mathescape=true,
     escapeinside={\%*}{*)},
     morekeywords={
        begin,
        end,
        if,
        then,
        else,
        to,
        endif,
        Procedure,
        while,
        do,
        return,
        true,
        false,
        set,
        for,
        repeat,
        until
        foreach},
 }

 \lstset{style=mystyle}

\renewcommand{\lstlistlistingname}{Elenco dei codici}
\renewcommand{\lstlistingname}{Codice}

\newtcolorbox{definition-box}[1]{%
    colback=codepurple!5!white,%
    colframe=codepurple!75!black,%
    fonttitle=\bfseries,%
    title={#1}
}

% mostra le subsubsection nell'indice
\setcounter{tocdepth}{3}
\setcounter{secnumdepth}{3}

% Resetta la numerazione dei chapter quando
% una nuova part viene creata
\makeatletter
\@addtoreset{chapter}{part}
\makeatother

% Rimuove l'indentazione quando si crea un nuovo paragrafo
\setlength{\parindent}{0pt}

% footer
\pagestyle{fancyplain}
% rimuove la riga nell'header
\fancyhf{} % sets both header and footer to nothing
\renewcommand{\headrulewidth}{0pt}
\fancyfoot[L]{\href{https://github.com/Typing-Monkeys/AppuntiUniversita}{Typing Monkeys}}
\fancyfoot[C]{\emoji{gorilla}}
\fancyfoot[R]{\thepage}

% configurazione emoji
\usepackage{fontspec}
\usepackage{emoji}
\setemojifont{NotoColorEmoji.ttf}[Path=/usr/share/fonts/truetype/noto/]

\setlength {\marginparwidth }{2cm}
\usepackage{todonotes}
\newcommand{\TODO}[2][]
{\todo[size=\scriptsize, color=red, #1]{#2}}

\newcommand{\properties}[1]{\textit{\textbf{Proprietà: }} #1 \vspace{5pt}}
\newcommand{\example}[1]{\textbf{\underline{Esempio:}}  #1}


\begin{document}

\include{frontmatter/main.tex}

%% Aggiungere i capitoli qui sotto
\include{capitoli/introduzione/main}
\include{capitoli/agenti-intelligenti/main}


\end{document}
\documentclass[a4paper,12 pt]{report}
\usepackage[T1]{fontenc}
\usepackage[utf8]{inputenc}
\usepackage[italian]{babel}
\usepackage{lmodern}
\usepackage{listings}
\usepackage{latexsym}
\usepackage{graphicx}
\usepackage{float}
\usepackage{subcaption}
\usepackage{hyperref}
\usepackage{wrapfig}
\usepackage{fancyhdr}
\usepackage{amsthm}
\usepackage{amsmath}
\usepackage{amssymb}
\usepackage{amsfonts}
\usepackage{cancel}
\usepackage{tcolorbox}


% forza le footnote a stare il più in basso possibile
\usepackage[bottom]{footmisc}

% stile teoremi
\newtheorem{theorem}{Teorema}[section]
\newtheorem{corollary}{Corollario}[theorem]
\newtheorem{lemma}[theorem]{Lemma}
\newtheorem{axch}{Assioma}[section]
\newtheorem{prop}{Proprietà}[section]
\newtheorem{definition}{Definizione}[section]
\newtheorem{limit}{Limitazione}[section]

\renewcommand*{\proofname}{Dimostrazione}

%% STILE LISTINGS

\usepackage{xcolor}

\definecolor{codegreen}{rgb}{0,0.6,0}
\definecolor{codegray}{rgb}{0.5,0.5,0.5}
\definecolor{codepurple}{rgb}{0.58,0,0.82}
\definecolor{backcolour}{rgb}{0.95,0.95,0.92}

 \lstdefinestyle{mystyle}{
     backgroundcolor=\color{backcolour},   
     commentstyle=\color{codegreen},
     keywordstyle=\bfseries\color{black},
     numberstyle=\tiny\color{codegray},
     stringstyle=\color{codepurple},
     basicstyle=\ttfamily\footnotesize,
     breakatwhitespace=false,         
     breaklines=true,                 
     captionpos=b,                    
     keepspaces=true,                 
     numbers=left,                    
     numbersep=5pt,                  
     showspaces=false,                
     showstringspaces=false,
     showtabs=false,                  
     tabsize=2,
     mathescape=true,
     escapeinside={\%*}{*)},
     morekeywords={
        begin,
        end,
        if,
        then,
        else,
        to,
        endif,
        Procedure,
        while,
        do,
        return,
        true,
        false,
        set,
        for,
        repeat,
        until
        foreach},
 }

 \lstset{style=mystyle}

\renewcommand{\lstlistlistingname}{Elenco dei codici}
\renewcommand{\lstlistingname}{Codice}

\newtcolorbox{definition-box}[1]{%
    colback=codepurple!5!white,%
    colframe=codepurple!75!black,%
    fonttitle=\bfseries,%
    title={#1}
}

% mostra le subsubsection nell'indice
\setcounter{tocdepth}{3}
\setcounter{secnumdepth}{3}

% Resetta la numerazione dei chapter quando
% una nuova part viene creata
\makeatletter
\@addtoreset{chapter}{part}
\makeatother

% Rimuove l'indentazione quando si crea un nuovo paragrafo
\setlength{\parindent}{0pt}

% footer
\pagestyle{fancyplain}
% rimuove la riga nell'header
\fancyhf{} % sets both header and footer to nothing
\renewcommand{\headrulewidth}{0pt}
\fancyfoot[L]{\href{https://github.com/Typing-Monkeys/AppuntiUniversita}{Typing Monkeys}}
\fancyfoot[C]{\emoji{gorilla}}
\fancyfoot[R]{\thepage}

% configurazione emoji
\usepackage{fontspec}
\usepackage{emoji}
\setemojifont{NotoColorEmoji.ttf}[Path=/usr/share/fonts/truetype/noto/]

\setlength {\marginparwidth }{2cm}
\usepackage{todonotes}
\newcommand{\TODO}[2][]
{\todo[size=\scriptsize, color=red, #1]{#2}}

\newcommand{\properties}[1]{\textit{\textbf{Proprietà: }} #1 \vspace{5pt}}
\newcommand{\example}[1]{\textbf{\underline{Esempio:}}  #1}


\begin{document}

\include{frontmatter/main.tex}

%% Aggiungere i capitoli qui sotto
\include{capitoli/introduzione/main}
\include{capitoli/agenti-intelligenti/main}


\end{document}


\end{document}


\end{document}
\documentclass[a4paper,12 pt]{report}
\usepackage[T1]{fontenc}
\usepackage[utf8]{inputenc}
\usepackage[italian]{babel}
\usepackage{lmodern}
\usepackage{listings}
\usepackage{latexsym}
\usepackage{graphicx}
\usepackage{float}
\usepackage{subcaption}
\usepackage{hyperref}
\usepackage{wrapfig}
\usepackage{fancyhdr}
\usepackage{amsthm}
\usepackage{amsmath}
\usepackage{amssymb}
\usepackage{amsfonts}
\usepackage{cancel}
\usepackage{tcolorbox}


% forza le footnote a stare il più in basso possibile
\usepackage[bottom]{footmisc}

% stile teoremi
\newtheorem{theorem}{Teorema}[section]
\newtheorem{corollary}{Corollario}[theorem]
\newtheorem{lemma}[theorem]{Lemma}
\newtheorem{axch}{Assioma}[section]
\newtheorem{prop}{Proprietà}[section]
\newtheorem{definition}{Definizione}[section]
\newtheorem{limit}{Limitazione}[section]

\renewcommand*{\proofname}{Dimostrazione}

%% STILE LISTINGS

\usepackage{xcolor}

\definecolor{codegreen}{rgb}{0,0.6,0}
\definecolor{codegray}{rgb}{0.5,0.5,0.5}
\definecolor{codepurple}{rgb}{0.58,0,0.82}
\definecolor{backcolour}{rgb}{0.95,0.95,0.92}

 \lstdefinestyle{mystyle}{
     backgroundcolor=\color{backcolour},   
     commentstyle=\color{codegreen},
     keywordstyle=\bfseries\color{black},
     numberstyle=\tiny\color{codegray},
     stringstyle=\color{codepurple},
     basicstyle=\ttfamily\footnotesize,
     breakatwhitespace=false,         
     breaklines=true,                 
     captionpos=b,                    
     keepspaces=true,                 
     numbers=left,                    
     numbersep=5pt,                  
     showspaces=false,                
     showstringspaces=false,
     showtabs=false,                  
     tabsize=2,
     mathescape=true,
     escapeinside={\%*}{*)},
     morekeywords={
        begin,
        end,
        if,
        then,
        else,
        to,
        endif,
        Procedure,
        while,
        do,
        return,
        true,
        false,
        set,
        for,
        repeat,
        until
        foreach},
 }

 \lstset{style=mystyle}

\renewcommand{\lstlistlistingname}{Elenco dei codici}
\renewcommand{\lstlistingname}{Codice}

\newtcolorbox{definition-box}[1]{%
    colback=codepurple!5!white,%
    colframe=codepurple!75!black,%
    fonttitle=\bfseries,%
    title={#1}
}

% mostra le subsubsection nell'indice
\setcounter{tocdepth}{3}
\setcounter{secnumdepth}{3}

% Resetta la numerazione dei chapter quando
% una nuova part viene creata
\makeatletter
\@addtoreset{chapter}{part}
\makeatother

% Rimuove l'indentazione quando si crea un nuovo paragrafo
\setlength{\parindent}{0pt}

% footer
\pagestyle{fancyplain}
% rimuove la riga nell'header
\fancyhf{} % sets both header and footer to nothing
\renewcommand{\headrulewidth}{0pt}
\fancyfoot[L]{\href{https://github.com/Typing-Monkeys/AppuntiUniversita}{Typing Monkeys}}
\fancyfoot[C]{\emoji{gorilla}}
\fancyfoot[R]{\thepage}

% configurazione emoji
\usepackage{fontspec}
\usepackage{emoji}
\setemojifont{NotoColorEmoji.ttf}[Path=/usr/share/fonts/truetype/noto/]

\setlength {\marginparwidth }{2cm}
\usepackage{todonotes}
\newcommand{\TODO}[2][]
{\todo[size=\scriptsize, color=red, #1]{#2}}

\newcommand{\properties}[1]{\textit{\textbf{Proprietà: }} #1 \vspace{5pt}}
\newcommand{\example}[1]{\textbf{\underline{Esempio:}}  #1}


\begin{document}

\documentclass[a4paper,12 pt]{report}
\usepackage[T1]{fontenc}
\usepackage[utf8]{inputenc}
\usepackage[italian]{babel}
\usepackage{lmodern}
\usepackage{listings}
\usepackage{latexsym}
\usepackage{graphicx}
\usepackage{float}
\usepackage{subcaption}
\usepackage{hyperref}
\usepackage{wrapfig}
\usepackage{fancyhdr}
\usepackage{amsthm}
\usepackage{amsmath}
\usepackage{amssymb}
\usepackage{amsfonts}
\usepackage{cancel}
\usepackage{tcolorbox}


% forza le footnote a stare il più in basso possibile
\usepackage[bottom]{footmisc}

% stile teoremi
\newtheorem{theorem}{Teorema}[section]
\newtheorem{corollary}{Corollario}[theorem]
\newtheorem{lemma}[theorem]{Lemma}
\newtheorem{axch}{Assioma}[section]
\newtheorem{prop}{Proprietà}[section]
\newtheorem{definition}{Definizione}[section]
\newtheorem{limit}{Limitazione}[section]

\renewcommand*{\proofname}{Dimostrazione}

%% STILE LISTINGS

\usepackage{xcolor}

\definecolor{codegreen}{rgb}{0,0.6,0}
\definecolor{codegray}{rgb}{0.5,0.5,0.5}
\definecolor{codepurple}{rgb}{0.58,0,0.82}
\definecolor{backcolour}{rgb}{0.95,0.95,0.92}

 \lstdefinestyle{mystyle}{
     backgroundcolor=\color{backcolour},   
     commentstyle=\color{codegreen},
     keywordstyle=\bfseries\color{black},
     numberstyle=\tiny\color{codegray},
     stringstyle=\color{codepurple},
     basicstyle=\ttfamily\footnotesize,
     breakatwhitespace=false,         
     breaklines=true,                 
     captionpos=b,                    
     keepspaces=true,                 
     numbers=left,                    
     numbersep=5pt,                  
     showspaces=false,                
     showstringspaces=false,
     showtabs=false,                  
     tabsize=2,
     mathescape=true,
     escapeinside={\%*}{*)},
     morekeywords={
        begin,
        end,
        if,
        then,
        else,
        to,
        endif,
        Procedure,
        while,
        do,
        return,
        true,
        false,
        set,
        for,
        repeat,
        until
        foreach},
 }

 \lstset{style=mystyle}

\renewcommand{\lstlistlistingname}{Elenco dei codici}
\renewcommand{\lstlistingname}{Codice}

\newtcolorbox{definition-box}[1]{%
    colback=codepurple!5!white,%
    colframe=codepurple!75!black,%
    fonttitle=\bfseries,%
    title={#1}
}

% mostra le subsubsection nell'indice
\setcounter{tocdepth}{3}
\setcounter{secnumdepth}{3}

% Resetta la numerazione dei chapter quando
% una nuova part viene creata
\makeatletter
\@addtoreset{chapter}{part}
\makeatother

% Rimuove l'indentazione quando si crea un nuovo paragrafo
\setlength{\parindent}{0pt}

% footer
\pagestyle{fancyplain}
% rimuove la riga nell'header
\fancyhf{} % sets both header and footer to nothing
\renewcommand{\headrulewidth}{0pt}
\fancyfoot[L]{\href{https://github.com/Typing-Monkeys/AppuntiUniversita}{Typing Monkeys}}
\fancyfoot[C]{\emoji{gorilla}}
\fancyfoot[R]{\thepage}

% configurazione emoji
\usepackage{fontspec}
\usepackage{emoji}
\setemojifont{NotoColorEmoji.ttf}[Path=/usr/share/fonts/truetype/noto/]

\setlength {\marginparwidth }{2cm}
\usepackage{todonotes}
\newcommand{\TODO}[2][]
{\todo[size=\scriptsize, color=red, #1]{#2}}

\newcommand{\properties}[1]{\textit{\textbf{Proprietà: }} #1 \vspace{5pt}}
\newcommand{\example}[1]{\textbf{\underline{Esempio:}}  #1}


\begin{document}

\documentclass[a4paper,12 pt]{report}
\usepackage[T1]{fontenc}
\usepackage[utf8]{inputenc}
\usepackage[italian]{babel}
\usepackage{lmodern}
\usepackage{listings}
\usepackage{latexsym}
\usepackage{graphicx}
\usepackage{float}
\usepackage{subcaption}
\usepackage{hyperref}
\usepackage{wrapfig}
\usepackage{fancyhdr}
\usepackage{amsthm}
\usepackage{amsmath}
\usepackage{amssymb}
\usepackage{amsfonts}
\usepackage{cancel}
\usepackage{tcolorbox}


% forza le footnote a stare il più in basso possibile
\usepackage[bottom]{footmisc}

% stile teoremi
\newtheorem{theorem}{Teorema}[section]
\newtheorem{corollary}{Corollario}[theorem]
\newtheorem{lemma}[theorem]{Lemma}
\newtheorem{axch}{Assioma}[section]
\newtheorem{prop}{Proprietà}[section]
\newtheorem{definition}{Definizione}[section]
\newtheorem{limit}{Limitazione}[section]

\renewcommand*{\proofname}{Dimostrazione}

%% STILE LISTINGS

\usepackage{xcolor}

\definecolor{codegreen}{rgb}{0,0.6,0}
\definecolor{codegray}{rgb}{0.5,0.5,0.5}
\definecolor{codepurple}{rgb}{0.58,0,0.82}
\definecolor{backcolour}{rgb}{0.95,0.95,0.92}

 \lstdefinestyle{mystyle}{
     backgroundcolor=\color{backcolour},   
     commentstyle=\color{codegreen},
     keywordstyle=\bfseries\color{black},
     numberstyle=\tiny\color{codegray},
     stringstyle=\color{codepurple},
     basicstyle=\ttfamily\footnotesize,
     breakatwhitespace=false,         
     breaklines=true,                 
     captionpos=b,                    
     keepspaces=true,                 
     numbers=left,                    
     numbersep=5pt,                  
     showspaces=false,                
     showstringspaces=false,
     showtabs=false,                  
     tabsize=2,
     mathescape=true,
     escapeinside={\%*}{*)},
     morekeywords={
        begin,
        end,
        if,
        then,
        else,
        to,
        endif,
        Procedure,
        while,
        do,
        return,
        true,
        false,
        set,
        for,
        repeat,
        until
        foreach},
 }

 \lstset{style=mystyle}

\renewcommand{\lstlistlistingname}{Elenco dei codici}
\renewcommand{\lstlistingname}{Codice}

\newtcolorbox{definition-box}[1]{%
    colback=codepurple!5!white,%
    colframe=codepurple!75!black,%
    fonttitle=\bfseries,%
    title={#1}
}

% mostra le subsubsection nell'indice
\setcounter{tocdepth}{3}
\setcounter{secnumdepth}{3}

% Resetta la numerazione dei chapter quando
% una nuova part viene creata
\makeatletter
\@addtoreset{chapter}{part}
\makeatother

% Rimuove l'indentazione quando si crea un nuovo paragrafo
\setlength{\parindent}{0pt}

% footer
\pagestyle{fancyplain}
% rimuove la riga nell'header
\fancyhf{} % sets both header and footer to nothing
\renewcommand{\headrulewidth}{0pt}
\fancyfoot[L]{\href{https://github.com/Typing-Monkeys/AppuntiUniversita}{Typing Monkeys}}
\fancyfoot[C]{\emoji{gorilla}}
\fancyfoot[R]{\thepage}

% configurazione emoji
\usepackage{fontspec}
\usepackage{emoji}
\setemojifont{NotoColorEmoji.ttf}[Path=/usr/share/fonts/truetype/noto/]

\setlength {\marginparwidth }{2cm}
\usepackage{todonotes}
\newcommand{\TODO}[2][]
{\todo[size=\scriptsize, color=red, #1]{#2}}

\newcommand{\properties}[1]{\textit{\textbf{Proprietà: }} #1 \vspace{5pt}}
\newcommand{\example}[1]{\textbf{\underline{Esempio:}}  #1}


\begin{document}

\include{frontmatter/main.tex}

%% Aggiungere i capitoli qui sotto
\include{capitoli/introduzione/main}
\include{capitoli/agenti-intelligenti/main}


\end{document}

%% Aggiungere i capitoli qui sotto
\documentclass[a4paper,12 pt]{report}
\usepackage[T1]{fontenc}
\usepackage[utf8]{inputenc}
\usepackage[italian]{babel}
\usepackage{lmodern}
\usepackage{listings}
\usepackage{latexsym}
\usepackage{graphicx}
\usepackage{float}
\usepackage{subcaption}
\usepackage{hyperref}
\usepackage{wrapfig}
\usepackage{fancyhdr}
\usepackage{amsthm}
\usepackage{amsmath}
\usepackage{amssymb}
\usepackage{amsfonts}
\usepackage{cancel}
\usepackage{tcolorbox}


% forza le footnote a stare il più in basso possibile
\usepackage[bottom]{footmisc}

% stile teoremi
\newtheorem{theorem}{Teorema}[section]
\newtheorem{corollary}{Corollario}[theorem]
\newtheorem{lemma}[theorem]{Lemma}
\newtheorem{axch}{Assioma}[section]
\newtheorem{prop}{Proprietà}[section]
\newtheorem{definition}{Definizione}[section]
\newtheorem{limit}{Limitazione}[section]

\renewcommand*{\proofname}{Dimostrazione}

%% STILE LISTINGS

\usepackage{xcolor}

\definecolor{codegreen}{rgb}{0,0.6,0}
\definecolor{codegray}{rgb}{0.5,0.5,0.5}
\definecolor{codepurple}{rgb}{0.58,0,0.82}
\definecolor{backcolour}{rgb}{0.95,0.95,0.92}

 \lstdefinestyle{mystyle}{
     backgroundcolor=\color{backcolour},   
     commentstyle=\color{codegreen},
     keywordstyle=\bfseries\color{black},
     numberstyle=\tiny\color{codegray},
     stringstyle=\color{codepurple},
     basicstyle=\ttfamily\footnotesize,
     breakatwhitespace=false,         
     breaklines=true,                 
     captionpos=b,                    
     keepspaces=true,                 
     numbers=left,                    
     numbersep=5pt,                  
     showspaces=false,                
     showstringspaces=false,
     showtabs=false,                  
     tabsize=2,
     mathescape=true,
     escapeinside={\%*}{*)},
     morekeywords={
        begin,
        end,
        if,
        then,
        else,
        to,
        endif,
        Procedure,
        while,
        do,
        return,
        true,
        false,
        set,
        for,
        repeat,
        until
        foreach},
 }

 \lstset{style=mystyle}

\renewcommand{\lstlistlistingname}{Elenco dei codici}
\renewcommand{\lstlistingname}{Codice}

\newtcolorbox{definition-box}[1]{%
    colback=codepurple!5!white,%
    colframe=codepurple!75!black,%
    fonttitle=\bfseries,%
    title={#1}
}

% mostra le subsubsection nell'indice
\setcounter{tocdepth}{3}
\setcounter{secnumdepth}{3}

% Resetta la numerazione dei chapter quando
% una nuova part viene creata
\makeatletter
\@addtoreset{chapter}{part}
\makeatother

% Rimuove l'indentazione quando si crea un nuovo paragrafo
\setlength{\parindent}{0pt}

% footer
\pagestyle{fancyplain}
% rimuove la riga nell'header
\fancyhf{} % sets both header and footer to nothing
\renewcommand{\headrulewidth}{0pt}
\fancyfoot[L]{\href{https://github.com/Typing-Monkeys/AppuntiUniversita}{Typing Monkeys}}
\fancyfoot[C]{\emoji{gorilla}}
\fancyfoot[R]{\thepage}

% configurazione emoji
\usepackage{fontspec}
\usepackage{emoji}
\setemojifont{NotoColorEmoji.ttf}[Path=/usr/share/fonts/truetype/noto/]

\setlength {\marginparwidth }{2cm}
\usepackage{todonotes}
\newcommand{\TODO}[2][]
{\todo[size=\scriptsize, color=red, #1]{#2}}

\newcommand{\properties}[1]{\textit{\textbf{Proprietà: }} #1 \vspace{5pt}}
\newcommand{\example}[1]{\textbf{\underline{Esempio:}}  #1}


\begin{document}

\include{frontmatter/main.tex}

%% Aggiungere i capitoli qui sotto
\include{capitoli/introduzione/main}
\include{capitoli/agenti-intelligenti/main}


\end{document}
\documentclass[a4paper,12 pt]{report}
\usepackage[T1]{fontenc}
\usepackage[utf8]{inputenc}
\usepackage[italian]{babel}
\usepackage{lmodern}
\usepackage{listings}
\usepackage{latexsym}
\usepackage{graphicx}
\usepackage{float}
\usepackage{subcaption}
\usepackage{hyperref}
\usepackage{wrapfig}
\usepackage{fancyhdr}
\usepackage{amsthm}
\usepackage{amsmath}
\usepackage{amssymb}
\usepackage{amsfonts}
\usepackage{cancel}
\usepackage{tcolorbox}


% forza le footnote a stare il più in basso possibile
\usepackage[bottom]{footmisc}

% stile teoremi
\newtheorem{theorem}{Teorema}[section]
\newtheorem{corollary}{Corollario}[theorem]
\newtheorem{lemma}[theorem]{Lemma}
\newtheorem{axch}{Assioma}[section]
\newtheorem{prop}{Proprietà}[section]
\newtheorem{definition}{Definizione}[section]
\newtheorem{limit}{Limitazione}[section]

\renewcommand*{\proofname}{Dimostrazione}

%% STILE LISTINGS

\usepackage{xcolor}

\definecolor{codegreen}{rgb}{0,0.6,0}
\definecolor{codegray}{rgb}{0.5,0.5,0.5}
\definecolor{codepurple}{rgb}{0.58,0,0.82}
\definecolor{backcolour}{rgb}{0.95,0.95,0.92}

 \lstdefinestyle{mystyle}{
     backgroundcolor=\color{backcolour},   
     commentstyle=\color{codegreen},
     keywordstyle=\bfseries\color{black},
     numberstyle=\tiny\color{codegray},
     stringstyle=\color{codepurple},
     basicstyle=\ttfamily\footnotesize,
     breakatwhitespace=false,         
     breaklines=true,                 
     captionpos=b,                    
     keepspaces=true,                 
     numbers=left,                    
     numbersep=5pt,                  
     showspaces=false,                
     showstringspaces=false,
     showtabs=false,                  
     tabsize=2,
     mathescape=true,
     escapeinside={\%*}{*)},
     morekeywords={
        begin,
        end,
        if,
        then,
        else,
        to,
        endif,
        Procedure,
        while,
        do,
        return,
        true,
        false,
        set,
        for,
        repeat,
        until
        foreach},
 }

 \lstset{style=mystyle}

\renewcommand{\lstlistlistingname}{Elenco dei codici}
\renewcommand{\lstlistingname}{Codice}

\newtcolorbox{definition-box}[1]{%
    colback=codepurple!5!white,%
    colframe=codepurple!75!black,%
    fonttitle=\bfseries,%
    title={#1}
}

% mostra le subsubsection nell'indice
\setcounter{tocdepth}{3}
\setcounter{secnumdepth}{3}

% Resetta la numerazione dei chapter quando
% una nuova part viene creata
\makeatletter
\@addtoreset{chapter}{part}
\makeatother

% Rimuove l'indentazione quando si crea un nuovo paragrafo
\setlength{\parindent}{0pt}

% footer
\pagestyle{fancyplain}
% rimuove la riga nell'header
\fancyhf{} % sets both header and footer to nothing
\renewcommand{\headrulewidth}{0pt}
\fancyfoot[L]{\href{https://github.com/Typing-Monkeys/AppuntiUniversita}{Typing Monkeys}}
\fancyfoot[C]{\emoji{gorilla}}
\fancyfoot[R]{\thepage}

% configurazione emoji
\usepackage{fontspec}
\usepackage{emoji}
\setemojifont{NotoColorEmoji.ttf}[Path=/usr/share/fonts/truetype/noto/]

\setlength {\marginparwidth }{2cm}
\usepackage{todonotes}
\newcommand{\TODO}[2][]
{\todo[size=\scriptsize, color=red, #1]{#2}}

\newcommand{\properties}[1]{\textit{\textbf{Proprietà: }} #1 \vspace{5pt}}
\newcommand{\example}[1]{\textbf{\underline{Esempio:}}  #1}


\begin{document}

\include{frontmatter/main.tex}

%% Aggiungere i capitoli qui sotto
\include{capitoli/introduzione/main}
\include{capitoli/agenti-intelligenti/main}


\end{document}


\end{document}

%% Aggiungere i capitoli qui sotto
\documentclass[a4paper,12 pt]{report}
\usepackage[T1]{fontenc}
\usepackage[utf8]{inputenc}
\usepackage[italian]{babel}
\usepackage{lmodern}
\usepackage{listings}
\usepackage{latexsym}
\usepackage{graphicx}
\usepackage{float}
\usepackage{subcaption}
\usepackage{hyperref}
\usepackage{wrapfig}
\usepackage{fancyhdr}
\usepackage{amsthm}
\usepackage{amsmath}
\usepackage{amssymb}
\usepackage{amsfonts}
\usepackage{cancel}
\usepackage{tcolorbox}


% forza le footnote a stare il più in basso possibile
\usepackage[bottom]{footmisc}

% stile teoremi
\newtheorem{theorem}{Teorema}[section]
\newtheorem{corollary}{Corollario}[theorem]
\newtheorem{lemma}[theorem]{Lemma}
\newtheorem{axch}{Assioma}[section]
\newtheorem{prop}{Proprietà}[section]
\newtheorem{definition}{Definizione}[section]
\newtheorem{limit}{Limitazione}[section]

\renewcommand*{\proofname}{Dimostrazione}

%% STILE LISTINGS

\usepackage{xcolor}

\definecolor{codegreen}{rgb}{0,0.6,0}
\definecolor{codegray}{rgb}{0.5,0.5,0.5}
\definecolor{codepurple}{rgb}{0.58,0,0.82}
\definecolor{backcolour}{rgb}{0.95,0.95,0.92}

 \lstdefinestyle{mystyle}{
     backgroundcolor=\color{backcolour},   
     commentstyle=\color{codegreen},
     keywordstyle=\bfseries\color{black},
     numberstyle=\tiny\color{codegray},
     stringstyle=\color{codepurple},
     basicstyle=\ttfamily\footnotesize,
     breakatwhitespace=false,         
     breaklines=true,                 
     captionpos=b,                    
     keepspaces=true,                 
     numbers=left,                    
     numbersep=5pt,                  
     showspaces=false,                
     showstringspaces=false,
     showtabs=false,                  
     tabsize=2,
     mathescape=true,
     escapeinside={\%*}{*)},
     morekeywords={
        begin,
        end,
        if,
        then,
        else,
        to,
        endif,
        Procedure,
        while,
        do,
        return,
        true,
        false,
        set,
        for,
        repeat,
        until
        foreach},
 }

 \lstset{style=mystyle}

\renewcommand{\lstlistlistingname}{Elenco dei codici}
\renewcommand{\lstlistingname}{Codice}

\newtcolorbox{definition-box}[1]{%
    colback=codepurple!5!white,%
    colframe=codepurple!75!black,%
    fonttitle=\bfseries,%
    title={#1}
}

% mostra le subsubsection nell'indice
\setcounter{tocdepth}{3}
\setcounter{secnumdepth}{3}

% Resetta la numerazione dei chapter quando
% una nuova part viene creata
\makeatletter
\@addtoreset{chapter}{part}
\makeatother

% Rimuove l'indentazione quando si crea un nuovo paragrafo
\setlength{\parindent}{0pt}

% footer
\pagestyle{fancyplain}
% rimuove la riga nell'header
\fancyhf{} % sets both header and footer to nothing
\renewcommand{\headrulewidth}{0pt}
\fancyfoot[L]{\href{https://github.com/Typing-Monkeys/AppuntiUniversita}{Typing Monkeys}}
\fancyfoot[C]{\emoji{gorilla}}
\fancyfoot[R]{\thepage}

% configurazione emoji
\usepackage{fontspec}
\usepackage{emoji}
\setemojifont{NotoColorEmoji.ttf}[Path=/usr/share/fonts/truetype/noto/]

\setlength {\marginparwidth }{2cm}
\usepackage{todonotes}
\newcommand{\TODO}[2][]
{\todo[size=\scriptsize, color=red, #1]{#2}}

\newcommand{\properties}[1]{\textit{\textbf{Proprietà: }} #1 \vspace{5pt}}
\newcommand{\example}[1]{\textbf{\underline{Esempio:}}  #1}


\begin{document}

\documentclass[a4paper,12 pt]{report}
\usepackage[T1]{fontenc}
\usepackage[utf8]{inputenc}
\usepackage[italian]{babel}
\usepackage{lmodern}
\usepackage{listings}
\usepackage{latexsym}
\usepackage{graphicx}
\usepackage{float}
\usepackage{subcaption}
\usepackage{hyperref}
\usepackage{wrapfig}
\usepackage{fancyhdr}
\usepackage{amsthm}
\usepackage{amsmath}
\usepackage{amssymb}
\usepackage{amsfonts}
\usepackage{cancel}
\usepackage{tcolorbox}


% forza le footnote a stare il più in basso possibile
\usepackage[bottom]{footmisc}

% stile teoremi
\newtheorem{theorem}{Teorema}[section]
\newtheorem{corollary}{Corollario}[theorem]
\newtheorem{lemma}[theorem]{Lemma}
\newtheorem{axch}{Assioma}[section]
\newtheorem{prop}{Proprietà}[section]
\newtheorem{definition}{Definizione}[section]
\newtheorem{limit}{Limitazione}[section]

\renewcommand*{\proofname}{Dimostrazione}

%% STILE LISTINGS

\usepackage{xcolor}

\definecolor{codegreen}{rgb}{0,0.6,0}
\definecolor{codegray}{rgb}{0.5,0.5,0.5}
\definecolor{codepurple}{rgb}{0.58,0,0.82}
\definecolor{backcolour}{rgb}{0.95,0.95,0.92}

 \lstdefinestyle{mystyle}{
     backgroundcolor=\color{backcolour},   
     commentstyle=\color{codegreen},
     keywordstyle=\bfseries\color{black},
     numberstyle=\tiny\color{codegray},
     stringstyle=\color{codepurple},
     basicstyle=\ttfamily\footnotesize,
     breakatwhitespace=false,         
     breaklines=true,                 
     captionpos=b,                    
     keepspaces=true,                 
     numbers=left,                    
     numbersep=5pt,                  
     showspaces=false,                
     showstringspaces=false,
     showtabs=false,                  
     tabsize=2,
     mathescape=true,
     escapeinside={\%*}{*)},
     morekeywords={
        begin,
        end,
        if,
        then,
        else,
        to,
        endif,
        Procedure,
        while,
        do,
        return,
        true,
        false,
        set,
        for,
        repeat,
        until
        foreach},
 }

 \lstset{style=mystyle}

\renewcommand{\lstlistlistingname}{Elenco dei codici}
\renewcommand{\lstlistingname}{Codice}

\newtcolorbox{definition-box}[1]{%
    colback=codepurple!5!white,%
    colframe=codepurple!75!black,%
    fonttitle=\bfseries,%
    title={#1}
}

% mostra le subsubsection nell'indice
\setcounter{tocdepth}{3}
\setcounter{secnumdepth}{3}

% Resetta la numerazione dei chapter quando
% una nuova part viene creata
\makeatletter
\@addtoreset{chapter}{part}
\makeatother

% Rimuove l'indentazione quando si crea un nuovo paragrafo
\setlength{\parindent}{0pt}

% footer
\pagestyle{fancyplain}
% rimuove la riga nell'header
\fancyhf{} % sets both header and footer to nothing
\renewcommand{\headrulewidth}{0pt}
\fancyfoot[L]{\href{https://github.com/Typing-Monkeys/AppuntiUniversita}{Typing Monkeys}}
\fancyfoot[C]{\emoji{gorilla}}
\fancyfoot[R]{\thepage}

% configurazione emoji
\usepackage{fontspec}
\usepackage{emoji}
\setemojifont{NotoColorEmoji.ttf}[Path=/usr/share/fonts/truetype/noto/]

\setlength {\marginparwidth }{2cm}
\usepackage{todonotes}
\newcommand{\TODO}[2][]
{\todo[size=\scriptsize, color=red, #1]{#2}}

\newcommand{\properties}[1]{\textit{\textbf{Proprietà: }} #1 \vspace{5pt}}
\newcommand{\example}[1]{\textbf{\underline{Esempio:}}  #1}


\begin{document}

\include{frontmatter/main.tex}

%% Aggiungere i capitoli qui sotto
\include{capitoli/introduzione/main}
\include{capitoli/agenti-intelligenti/main}


\end{document}

%% Aggiungere i capitoli qui sotto
\documentclass[a4paper,12 pt]{report}
\usepackage[T1]{fontenc}
\usepackage[utf8]{inputenc}
\usepackage[italian]{babel}
\usepackage{lmodern}
\usepackage{listings}
\usepackage{latexsym}
\usepackage{graphicx}
\usepackage{float}
\usepackage{subcaption}
\usepackage{hyperref}
\usepackage{wrapfig}
\usepackage{fancyhdr}
\usepackage{amsthm}
\usepackage{amsmath}
\usepackage{amssymb}
\usepackage{amsfonts}
\usepackage{cancel}
\usepackage{tcolorbox}


% forza le footnote a stare il più in basso possibile
\usepackage[bottom]{footmisc}

% stile teoremi
\newtheorem{theorem}{Teorema}[section]
\newtheorem{corollary}{Corollario}[theorem]
\newtheorem{lemma}[theorem]{Lemma}
\newtheorem{axch}{Assioma}[section]
\newtheorem{prop}{Proprietà}[section]
\newtheorem{definition}{Definizione}[section]
\newtheorem{limit}{Limitazione}[section]

\renewcommand*{\proofname}{Dimostrazione}

%% STILE LISTINGS

\usepackage{xcolor}

\definecolor{codegreen}{rgb}{0,0.6,0}
\definecolor{codegray}{rgb}{0.5,0.5,0.5}
\definecolor{codepurple}{rgb}{0.58,0,0.82}
\definecolor{backcolour}{rgb}{0.95,0.95,0.92}

 \lstdefinestyle{mystyle}{
     backgroundcolor=\color{backcolour},   
     commentstyle=\color{codegreen},
     keywordstyle=\bfseries\color{black},
     numberstyle=\tiny\color{codegray},
     stringstyle=\color{codepurple},
     basicstyle=\ttfamily\footnotesize,
     breakatwhitespace=false,         
     breaklines=true,                 
     captionpos=b,                    
     keepspaces=true,                 
     numbers=left,                    
     numbersep=5pt,                  
     showspaces=false,                
     showstringspaces=false,
     showtabs=false,                  
     tabsize=2,
     mathescape=true,
     escapeinside={\%*}{*)},
     morekeywords={
        begin,
        end,
        if,
        then,
        else,
        to,
        endif,
        Procedure,
        while,
        do,
        return,
        true,
        false,
        set,
        for,
        repeat,
        until
        foreach},
 }

 \lstset{style=mystyle}

\renewcommand{\lstlistlistingname}{Elenco dei codici}
\renewcommand{\lstlistingname}{Codice}

\newtcolorbox{definition-box}[1]{%
    colback=codepurple!5!white,%
    colframe=codepurple!75!black,%
    fonttitle=\bfseries,%
    title={#1}
}

% mostra le subsubsection nell'indice
\setcounter{tocdepth}{3}
\setcounter{secnumdepth}{3}

% Resetta la numerazione dei chapter quando
% una nuova part viene creata
\makeatletter
\@addtoreset{chapter}{part}
\makeatother

% Rimuove l'indentazione quando si crea un nuovo paragrafo
\setlength{\parindent}{0pt}

% footer
\pagestyle{fancyplain}
% rimuove la riga nell'header
\fancyhf{} % sets both header and footer to nothing
\renewcommand{\headrulewidth}{0pt}
\fancyfoot[L]{\href{https://github.com/Typing-Monkeys/AppuntiUniversita}{Typing Monkeys}}
\fancyfoot[C]{\emoji{gorilla}}
\fancyfoot[R]{\thepage}

% configurazione emoji
\usepackage{fontspec}
\usepackage{emoji}
\setemojifont{NotoColorEmoji.ttf}[Path=/usr/share/fonts/truetype/noto/]

\setlength {\marginparwidth }{2cm}
\usepackage{todonotes}
\newcommand{\TODO}[2][]
{\todo[size=\scriptsize, color=red, #1]{#2}}

\newcommand{\properties}[1]{\textit{\textbf{Proprietà: }} #1 \vspace{5pt}}
\newcommand{\example}[1]{\textbf{\underline{Esempio:}}  #1}


\begin{document}

\include{frontmatter/main.tex}

%% Aggiungere i capitoli qui sotto
\include{capitoli/introduzione/main}
\include{capitoli/agenti-intelligenti/main}


\end{document}
\documentclass[a4paper,12 pt]{report}
\usepackage[T1]{fontenc}
\usepackage[utf8]{inputenc}
\usepackage[italian]{babel}
\usepackage{lmodern}
\usepackage{listings}
\usepackage{latexsym}
\usepackage{graphicx}
\usepackage{float}
\usepackage{subcaption}
\usepackage{hyperref}
\usepackage{wrapfig}
\usepackage{fancyhdr}
\usepackage{amsthm}
\usepackage{amsmath}
\usepackage{amssymb}
\usepackage{amsfonts}
\usepackage{cancel}
\usepackage{tcolorbox}


% forza le footnote a stare il più in basso possibile
\usepackage[bottom]{footmisc}

% stile teoremi
\newtheorem{theorem}{Teorema}[section]
\newtheorem{corollary}{Corollario}[theorem]
\newtheorem{lemma}[theorem]{Lemma}
\newtheorem{axch}{Assioma}[section]
\newtheorem{prop}{Proprietà}[section]
\newtheorem{definition}{Definizione}[section]
\newtheorem{limit}{Limitazione}[section]

\renewcommand*{\proofname}{Dimostrazione}

%% STILE LISTINGS

\usepackage{xcolor}

\definecolor{codegreen}{rgb}{0,0.6,0}
\definecolor{codegray}{rgb}{0.5,0.5,0.5}
\definecolor{codepurple}{rgb}{0.58,0,0.82}
\definecolor{backcolour}{rgb}{0.95,0.95,0.92}

 \lstdefinestyle{mystyle}{
     backgroundcolor=\color{backcolour},   
     commentstyle=\color{codegreen},
     keywordstyle=\bfseries\color{black},
     numberstyle=\tiny\color{codegray},
     stringstyle=\color{codepurple},
     basicstyle=\ttfamily\footnotesize,
     breakatwhitespace=false,         
     breaklines=true,                 
     captionpos=b,                    
     keepspaces=true,                 
     numbers=left,                    
     numbersep=5pt,                  
     showspaces=false,                
     showstringspaces=false,
     showtabs=false,                  
     tabsize=2,
     mathescape=true,
     escapeinside={\%*}{*)},
     morekeywords={
        begin,
        end,
        if,
        then,
        else,
        to,
        endif,
        Procedure,
        while,
        do,
        return,
        true,
        false,
        set,
        for,
        repeat,
        until
        foreach},
 }

 \lstset{style=mystyle}

\renewcommand{\lstlistlistingname}{Elenco dei codici}
\renewcommand{\lstlistingname}{Codice}

\newtcolorbox{definition-box}[1]{%
    colback=codepurple!5!white,%
    colframe=codepurple!75!black,%
    fonttitle=\bfseries,%
    title={#1}
}

% mostra le subsubsection nell'indice
\setcounter{tocdepth}{3}
\setcounter{secnumdepth}{3}

% Resetta la numerazione dei chapter quando
% una nuova part viene creata
\makeatletter
\@addtoreset{chapter}{part}
\makeatother

% Rimuove l'indentazione quando si crea un nuovo paragrafo
\setlength{\parindent}{0pt}

% footer
\pagestyle{fancyplain}
% rimuove la riga nell'header
\fancyhf{} % sets both header and footer to nothing
\renewcommand{\headrulewidth}{0pt}
\fancyfoot[L]{\href{https://github.com/Typing-Monkeys/AppuntiUniversita}{Typing Monkeys}}
\fancyfoot[C]{\emoji{gorilla}}
\fancyfoot[R]{\thepage}

% configurazione emoji
\usepackage{fontspec}
\usepackage{emoji}
\setemojifont{NotoColorEmoji.ttf}[Path=/usr/share/fonts/truetype/noto/]

\setlength {\marginparwidth }{2cm}
\usepackage{todonotes}
\newcommand{\TODO}[2][]
{\todo[size=\scriptsize, color=red, #1]{#2}}

\newcommand{\properties}[1]{\textit{\textbf{Proprietà: }} #1 \vspace{5pt}}
\newcommand{\example}[1]{\textbf{\underline{Esempio:}}  #1}


\begin{document}

\include{frontmatter/main.tex}

%% Aggiungere i capitoli qui sotto
\include{capitoli/introduzione/main}
\include{capitoli/agenti-intelligenti/main}


\end{document}


\end{document}
\documentclass[a4paper,12 pt]{report}
\usepackage[T1]{fontenc}
\usepackage[utf8]{inputenc}
\usepackage[italian]{babel}
\usepackage{lmodern}
\usepackage{listings}
\usepackage{latexsym}
\usepackage{graphicx}
\usepackage{float}
\usepackage{subcaption}
\usepackage{hyperref}
\usepackage{wrapfig}
\usepackage{fancyhdr}
\usepackage{amsthm}
\usepackage{amsmath}
\usepackage{amssymb}
\usepackage{amsfonts}
\usepackage{cancel}
\usepackage{tcolorbox}


% forza le footnote a stare il più in basso possibile
\usepackage[bottom]{footmisc}

% stile teoremi
\newtheorem{theorem}{Teorema}[section]
\newtheorem{corollary}{Corollario}[theorem]
\newtheorem{lemma}[theorem]{Lemma}
\newtheorem{axch}{Assioma}[section]
\newtheorem{prop}{Proprietà}[section]
\newtheorem{definition}{Definizione}[section]
\newtheorem{limit}{Limitazione}[section]

\renewcommand*{\proofname}{Dimostrazione}

%% STILE LISTINGS

\usepackage{xcolor}

\definecolor{codegreen}{rgb}{0,0.6,0}
\definecolor{codegray}{rgb}{0.5,0.5,0.5}
\definecolor{codepurple}{rgb}{0.58,0,0.82}
\definecolor{backcolour}{rgb}{0.95,0.95,0.92}

 \lstdefinestyle{mystyle}{
     backgroundcolor=\color{backcolour},   
     commentstyle=\color{codegreen},
     keywordstyle=\bfseries\color{black},
     numberstyle=\tiny\color{codegray},
     stringstyle=\color{codepurple},
     basicstyle=\ttfamily\footnotesize,
     breakatwhitespace=false,         
     breaklines=true,                 
     captionpos=b,                    
     keepspaces=true,                 
     numbers=left,                    
     numbersep=5pt,                  
     showspaces=false,                
     showstringspaces=false,
     showtabs=false,                  
     tabsize=2,
     mathescape=true,
     escapeinside={\%*}{*)},
     morekeywords={
        begin,
        end,
        if,
        then,
        else,
        to,
        endif,
        Procedure,
        while,
        do,
        return,
        true,
        false,
        set,
        for,
        repeat,
        until
        foreach},
 }

 \lstset{style=mystyle}

\renewcommand{\lstlistlistingname}{Elenco dei codici}
\renewcommand{\lstlistingname}{Codice}

\newtcolorbox{definition-box}[1]{%
    colback=codepurple!5!white,%
    colframe=codepurple!75!black,%
    fonttitle=\bfseries,%
    title={#1}
}

% mostra le subsubsection nell'indice
\setcounter{tocdepth}{3}
\setcounter{secnumdepth}{3}

% Resetta la numerazione dei chapter quando
% una nuova part viene creata
\makeatletter
\@addtoreset{chapter}{part}
\makeatother

% Rimuove l'indentazione quando si crea un nuovo paragrafo
\setlength{\parindent}{0pt}

% footer
\pagestyle{fancyplain}
% rimuove la riga nell'header
\fancyhf{} % sets both header and footer to nothing
\renewcommand{\headrulewidth}{0pt}
\fancyfoot[L]{\href{https://github.com/Typing-Monkeys/AppuntiUniversita}{Typing Monkeys}}
\fancyfoot[C]{\emoji{gorilla}}
\fancyfoot[R]{\thepage}

% configurazione emoji
\usepackage{fontspec}
\usepackage{emoji}
\setemojifont{NotoColorEmoji.ttf}[Path=/usr/share/fonts/truetype/noto/]

\setlength {\marginparwidth }{2cm}
\usepackage{todonotes}
\newcommand{\TODO}[2][]
{\todo[size=\scriptsize, color=red, #1]{#2}}

\newcommand{\properties}[1]{\textit{\textbf{Proprietà: }} #1 \vspace{5pt}}
\newcommand{\example}[1]{\textbf{\underline{Esempio:}}  #1}


\begin{document}

\documentclass[a4paper,12 pt]{report}
\usepackage[T1]{fontenc}
\usepackage[utf8]{inputenc}
\usepackage[italian]{babel}
\usepackage{lmodern}
\usepackage{listings}
\usepackage{latexsym}
\usepackage{graphicx}
\usepackage{float}
\usepackage{subcaption}
\usepackage{hyperref}
\usepackage{wrapfig}
\usepackage{fancyhdr}
\usepackage{amsthm}
\usepackage{amsmath}
\usepackage{amssymb}
\usepackage{amsfonts}
\usepackage{cancel}
\usepackage{tcolorbox}


% forza le footnote a stare il più in basso possibile
\usepackage[bottom]{footmisc}

% stile teoremi
\newtheorem{theorem}{Teorema}[section]
\newtheorem{corollary}{Corollario}[theorem]
\newtheorem{lemma}[theorem]{Lemma}
\newtheorem{axch}{Assioma}[section]
\newtheorem{prop}{Proprietà}[section]
\newtheorem{definition}{Definizione}[section]
\newtheorem{limit}{Limitazione}[section]

\renewcommand*{\proofname}{Dimostrazione}

%% STILE LISTINGS

\usepackage{xcolor}

\definecolor{codegreen}{rgb}{0,0.6,0}
\definecolor{codegray}{rgb}{0.5,0.5,0.5}
\definecolor{codepurple}{rgb}{0.58,0,0.82}
\definecolor{backcolour}{rgb}{0.95,0.95,0.92}

 \lstdefinestyle{mystyle}{
     backgroundcolor=\color{backcolour},   
     commentstyle=\color{codegreen},
     keywordstyle=\bfseries\color{black},
     numberstyle=\tiny\color{codegray},
     stringstyle=\color{codepurple},
     basicstyle=\ttfamily\footnotesize,
     breakatwhitespace=false,         
     breaklines=true,                 
     captionpos=b,                    
     keepspaces=true,                 
     numbers=left,                    
     numbersep=5pt,                  
     showspaces=false,                
     showstringspaces=false,
     showtabs=false,                  
     tabsize=2,
     mathescape=true,
     escapeinside={\%*}{*)},
     morekeywords={
        begin,
        end,
        if,
        then,
        else,
        to,
        endif,
        Procedure,
        while,
        do,
        return,
        true,
        false,
        set,
        for,
        repeat,
        until
        foreach},
 }

 \lstset{style=mystyle}

\renewcommand{\lstlistlistingname}{Elenco dei codici}
\renewcommand{\lstlistingname}{Codice}

\newtcolorbox{definition-box}[1]{%
    colback=codepurple!5!white,%
    colframe=codepurple!75!black,%
    fonttitle=\bfseries,%
    title={#1}
}

% mostra le subsubsection nell'indice
\setcounter{tocdepth}{3}
\setcounter{secnumdepth}{3}

% Resetta la numerazione dei chapter quando
% una nuova part viene creata
\makeatletter
\@addtoreset{chapter}{part}
\makeatother

% Rimuove l'indentazione quando si crea un nuovo paragrafo
\setlength{\parindent}{0pt}

% footer
\pagestyle{fancyplain}
% rimuove la riga nell'header
\fancyhf{} % sets both header and footer to nothing
\renewcommand{\headrulewidth}{0pt}
\fancyfoot[L]{\href{https://github.com/Typing-Monkeys/AppuntiUniversita}{Typing Monkeys}}
\fancyfoot[C]{\emoji{gorilla}}
\fancyfoot[R]{\thepage}

% configurazione emoji
\usepackage{fontspec}
\usepackage{emoji}
\setemojifont{NotoColorEmoji.ttf}[Path=/usr/share/fonts/truetype/noto/]

\setlength {\marginparwidth }{2cm}
\usepackage{todonotes}
\newcommand{\TODO}[2][]
{\todo[size=\scriptsize, color=red, #1]{#2}}

\newcommand{\properties}[1]{\textit{\textbf{Proprietà: }} #1 \vspace{5pt}}
\newcommand{\example}[1]{\textbf{\underline{Esempio:}}  #1}


\begin{document}

\include{frontmatter/main.tex}

%% Aggiungere i capitoli qui sotto
\include{capitoli/introduzione/main}
\include{capitoli/agenti-intelligenti/main}


\end{document}

%% Aggiungere i capitoli qui sotto
\documentclass[a4paper,12 pt]{report}
\usepackage[T1]{fontenc}
\usepackage[utf8]{inputenc}
\usepackage[italian]{babel}
\usepackage{lmodern}
\usepackage{listings}
\usepackage{latexsym}
\usepackage{graphicx}
\usepackage{float}
\usepackage{subcaption}
\usepackage{hyperref}
\usepackage{wrapfig}
\usepackage{fancyhdr}
\usepackage{amsthm}
\usepackage{amsmath}
\usepackage{amssymb}
\usepackage{amsfonts}
\usepackage{cancel}
\usepackage{tcolorbox}


% forza le footnote a stare il più in basso possibile
\usepackage[bottom]{footmisc}

% stile teoremi
\newtheorem{theorem}{Teorema}[section]
\newtheorem{corollary}{Corollario}[theorem]
\newtheorem{lemma}[theorem]{Lemma}
\newtheorem{axch}{Assioma}[section]
\newtheorem{prop}{Proprietà}[section]
\newtheorem{definition}{Definizione}[section]
\newtheorem{limit}{Limitazione}[section]

\renewcommand*{\proofname}{Dimostrazione}

%% STILE LISTINGS

\usepackage{xcolor}

\definecolor{codegreen}{rgb}{0,0.6,0}
\definecolor{codegray}{rgb}{0.5,0.5,0.5}
\definecolor{codepurple}{rgb}{0.58,0,0.82}
\definecolor{backcolour}{rgb}{0.95,0.95,0.92}

 \lstdefinestyle{mystyle}{
     backgroundcolor=\color{backcolour},   
     commentstyle=\color{codegreen},
     keywordstyle=\bfseries\color{black},
     numberstyle=\tiny\color{codegray},
     stringstyle=\color{codepurple},
     basicstyle=\ttfamily\footnotesize,
     breakatwhitespace=false,         
     breaklines=true,                 
     captionpos=b,                    
     keepspaces=true,                 
     numbers=left,                    
     numbersep=5pt,                  
     showspaces=false,                
     showstringspaces=false,
     showtabs=false,                  
     tabsize=2,
     mathescape=true,
     escapeinside={\%*}{*)},
     morekeywords={
        begin,
        end,
        if,
        then,
        else,
        to,
        endif,
        Procedure,
        while,
        do,
        return,
        true,
        false,
        set,
        for,
        repeat,
        until
        foreach},
 }

 \lstset{style=mystyle}

\renewcommand{\lstlistlistingname}{Elenco dei codici}
\renewcommand{\lstlistingname}{Codice}

\newtcolorbox{definition-box}[1]{%
    colback=codepurple!5!white,%
    colframe=codepurple!75!black,%
    fonttitle=\bfseries,%
    title={#1}
}

% mostra le subsubsection nell'indice
\setcounter{tocdepth}{3}
\setcounter{secnumdepth}{3}

% Resetta la numerazione dei chapter quando
% una nuova part viene creata
\makeatletter
\@addtoreset{chapter}{part}
\makeatother

% Rimuove l'indentazione quando si crea un nuovo paragrafo
\setlength{\parindent}{0pt}

% footer
\pagestyle{fancyplain}
% rimuove la riga nell'header
\fancyhf{} % sets both header and footer to nothing
\renewcommand{\headrulewidth}{0pt}
\fancyfoot[L]{\href{https://github.com/Typing-Monkeys/AppuntiUniversita}{Typing Monkeys}}
\fancyfoot[C]{\emoji{gorilla}}
\fancyfoot[R]{\thepage}

% configurazione emoji
\usepackage{fontspec}
\usepackage{emoji}
\setemojifont{NotoColorEmoji.ttf}[Path=/usr/share/fonts/truetype/noto/]

\setlength {\marginparwidth }{2cm}
\usepackage{todonotes}
\newcommand{\TODO}[2][]
{\todo[size=\scriptsize, color=red, #1]{#2}}

\newcommand{\properties}[1]{\textit{\textbf{Proprietà: }} #1 \vspace{5pt}}
\newcommand{\example}[1]{\textbf{\underline{Esempio:}}  #1}


\begin{document}

\include{frontmatter/main.tex}

%% Aggiungere i capitoli qui sotto
\include{capitoli/introduzione/main}
\include{capitoli/agenti-intelligenti/main}


\end{document}
\documentclass[a4paper,12 pt]{report}
\usepackage[T1]{fontenc}
\usepackage[utf8]{inputenc}
\usepackage[italian]{babel}
\usepackage{lmodern}
\usepackage{listings}
\usepackage{latexsym}
\usepackage{graphicx}
\usepackage{float}
\usepackage{subcaption}
\usepackage{hyperref}
\usepackage{wrapfig}
\usepackage{fancyhdr}
\usepackage{amsthm}
\usepackage{amsmath}
\usepackage{amssymb}
\usepackage{amsfonts}
\usepackage{cancel}
\usepackage{tcolorbox}


% forza le footnote a stare il più in basso possibile
\usepackage[bottom]{footmisc}

% stile teoremi
\newtheorem{theorem}{Teorema}[section]
\newtheorem{corollary}{Corollario}[theorem]
\newtheorem{lemma}[theorem]{Lemma}
\newtheorem{axch}{Assioma}[section]
\newtheorem{prop}{Proprietà}[section]
\newtheorem{definition}{Definizione}[section]
\newtheorem{limit}{Limitazione}[section]

\renewcommand*{\proofname}{Dimostrazione}

%% STILE LISTINGS

\usepackage{xcolor}

\definecolor{codegreen}{rgb}{0,0.6,0}
\definecolor{codegray}{rgb}{0.5,0.5,0.5}
\definecolor{codepurple}{rgb}{0.58,0,0.82}
\definecolor{backcolour}{rgb}{0.95,0.95,0.92}

 \lstdefinestyle{mystyle}{
     backgroundcolor=\color{backcolour},   
     commentstyle=\color{codegreen},
     keywordstyle=\bfseries\color{black},
     numberstyle=\tiny\color{codegray},
     stringstyle=\color{codepurple},
     basicstyle=\ttfamily\footnotesize,
     breakatwhitespace=false,         
     breaklines=true,                 
     captionpos=b,                    
     keepspaces=true,                 
     numbers=left,                    
     numbersep=5pt,                  
     showspaces=false,                
     showstringspaces=false,
     showtabs=false,                  
     tabsize=2,
     mathescape=true,
     escapeinside={\%*}{*)},
     morekeywords={
        begin,
        end,
        if,
        then,
        else,
        to,
        endif,
        Procedure,
        while,
        do,
        return,
        true,
        false,
        set,
        for,
        repeat,
        until
        foreach},
 }

 \lstset{style=mystyle}

\renewcommand{\lstlistlistingname}{Elenco dei codici}
\renewcommand{\lstlistingname}{Codice}

\newtcolorbox{definition-box}[1]{%
    colback=codepurple!5!white,%
    colframe=codepurple!75!black,%
    fonttitle=\bfseries,%
    title={#1}
}

% mostra le subsubsection nell'indice
\setcounter{tocdepth}{3}
\setcounter{secnumdepth}{3}

% Resetta la numerazione dei chapter quando
% una nuova part viene creata
\makeatletter
\@addtoreset{chapter}{part}
\makeatother

% Rimuove l'indentazione quando si crea un nuovo paragrafo
\setlength{\parindent}{0pt}

% footer
\pagestyle{fancyplain}
% rimuove la riga nell'header
\fancyhf{} % sets both header and footer to nothing
\renewcommand{\headrulewidth}{0pt}
\fancyfoot[L]{\href{https://github.com/Typing-Monkeys/AppuntiUniversita}{Typing Monkeys}}
\fancyfoot[C]{\emoji{gorilla}}
\fancyfoot[R]{\thepage}

% configurazione emoji
\usepackage{fontspec}
\usepackage{emoji}
\setemojifont{NotoColorEmoji.ttf}[Path=/usr/share/fonts/truetype/noto/]

\setlength {\marginparwidth }{2cm}
\usepackage{todonotes}
\newcommand{\TODO}[2][]
{\todo[size=\scriptsize, color=red, #1]{#2}}

\newcommand{\properties}[1]{\textit{\textbf{Proprietà: }} #1 \vspace{5pt}}
\newcommand{\example}[1]{\textbf{\underline{Esempio:}}  #1}


\begin{document}

\include{frontmatter/main.tex}

%% Aggiungere i capitoli qui sotto
\include{capitoli/introduzione/main}
\include{capitoli/agenti-intelligenti/main}


\end{document}


\end{document}


\end{document}


\end{document}